\begin{proposition}\label{prop:monotone}
    The dimension $k_{\kappa}$ of $\mathcal S_{\kappa}$ is nonincreasing in $\kappa$ for $\kappa>0$.
That is, $k_{\kappa_1} \geq k_{\kappa_2}$ for $ \kappa _2 > \kappa_1 > 0$.
\end{proposition}

\begin{proof}[\textbf{Proof of Proposition \ref{prop:monotone}}]
    For positive integer $m$, define $[m]=\{1,,\ldots, m\}$.
    For a set $A$, denote by $| A |$ its cardinality.
    We have
    \begin{equation*}
        \begin{split}
        k_{\kappa} =& \left|\left\{i\in [n-q]: \frac{\gamma_i^2}{\gamma_i +\kappa} - \frac{1}{n-q} \sum_{j=1}^{n-q}\frac{\gamma_j \gamma_i}{\gamma_j +\kappa}>0 \right\}\right|
        \\
        =& \left|\left\{i\in [n-q]: \frac{\gamma_i}{\gamma_i +\kappa} > \frac{1}{n-q} \sum_{j=1}^{n-q}\frac{\gamma_j }{\gamma_j +\kappa} \right\}\right|
        .
        \end{split}
    \end{equation*}
    Let $X$ be a random variable uniformly distributed on $\{\gamma_1,\ldots,\gamma_{n-q}\}$.
    That is, $\Pr(X=\gamma_i)=1/(n-q)$, $i=1,\ldots, n-q$.
    Then it can be seen that
    \begin{equation*}
        k_{\kappa}=(n-q) \Pr \left(\frac{X}{X+\kappa}>\myE \left[\frac{X}{X+\kappa}\right]\right).
    \end{equation*}
    Hence we only need to verify
    \begin{equation}\label{eq:toBeJen}
        \Pr \left(\frac{X}{X+\kappa_1}>\myE \left[\frac{X}{X+\kappa_1}\right]\right) 
\geq
\Pr \left(\frac{X}{X+\kappa_2}>\myE \left[\frac{X}{X+\kappa_2}\right]\right) .
    \end{equation}
    Let $Y=X/(X+\kappa_2)$.
    Then
    \begin{equation*}
        \frac{X}{(X+\kappa_1)} = \frac{\kappa_2 Y}{ \kappa_1 + (\kappa_2-\kappa_1) Y} =: f(Y).
    \end{equation*}
    Note that $f(Y)$ is increasing for $Y\geq 0$.
    Then the inequality \eqref{eq:toBeJen} is equivalent to
    \begin{equation*}
        \Pr \left(Y> f^{-1}\left(\myE f(Y)\right)\right) 
\geq
\Pr \left( Y >\myE Y\right) .
    \end{equation*}
    Hence we only need to verify
        $f^{-1}\left(\myE f(Y)\right)
        \leq
        \myE Y$, or equivalently, $\myE f(Y)
        \leq
        f(\myE Y)$.
        But the last inequality is a direct consequence of the concavity of $f(Y)$ and Jensen's inequality.
        This completes the proof.

\end{proof}
