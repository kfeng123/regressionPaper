\documentclass[11pt]{article}
 
\newcommand\CG[1]{\textcolor{red}{#1}}

\usepackage{lineno,hyperref}

\usepackage[margin=1 in]{geometry}
\renewcommand{\baselinestretch}{1.25}


%\usepackage{refcheck}
\usepackage{authblk}
\usepackage{galois} % composition function \comp
\usepackage{bm}
\usepackage{amsmath}
\usepackage{amssymb}
\usepackage{mathrsfs}
\usepackage{amsthm}
\usepackage{natbib}
\usepackage{graphicx}
\usepackage{color}
\usepackage{booktabs}
\usepackage[page,title]{appendix}
%\renewcommand\appendixname{haha}
\usepackage{enumerate}
\usepackage{changepage}
\usepackage{datetime}
\newdate{date}{9}{1}{2017}

%%%%%%%%%%%%%%  Notations %%%%%%%%%%
\DeclareMathOperator{\mytr}{tr}
\DeclareMathOperator{\mydiag}{diag}
\DeclareMathOperator{\myRank}{Rank}
\DeclareMathOperator{\myP}{P}
\DeclareMathOperator{\myE}{E}
\DeclareMathOperator{\myVar}{Var}
\DeclareMathOperator{\myCov}{Cov}
\DeclareMathOperator*{\argmax}{arg\,max}
\DeclareMathOperator*{\argmin}{arg\,min}


\newcommand{\Ba}{\mathbf{a}}    \newcommand{\Bb}{\mathbf{b}}    \newcommand{\Bc}{\mathbf{c}}    \newcommand{\Bd}{\mathbf{d}}    \newcommand{\Be}{\mathbf{e}}    \newcommand{\Bf}{\mathbf{f}}    \newcommand{\Bg}{\mathbf{g}}    \newcommand{\Bh}{\mathbf{h}}    \newcommand{\Bi}{\mathbf{i}}    \newcommand{\Bj}{\mathbf{j}}    \newcommand{\Bk}{\mathbf{k}}    \newcommand{\Bl}{\mathbf{l}}
\newcommand{\Bm}{\mathbf{m}}    \newcommand{\Bn}{\mathbf{n}}    \newcommand{\Bo}{\mathbf{o}}    \newcommand{\Bp}{\mathbf{p}}    \newcommand{\Bq}{\mathbf{q}}    \newcommand{\Br}{\mathbf{r}}    \newcommand{\Bs}{\mathbf{s}}    \newcommand{\Bt}{\mathbf{t}}    \newcommand{\Bu}{\mathbf{u}}    \newcommand{\Bv}{\mathbf{v}}    \newcommand{\Bw}{\mathbf{w}}    \newcommand{\Bx}{\mathbf{x}}
\newcommand{\By}{\mathbf{y}}    \newcommand{\Bz}{\mathbf{z}}    
\newcommand{\BA}{\mathbf{A}}    \newcommand{\BB}{\mathbf{B}}    \newcommand{\BC}{\mathbf{C}}    \newcommand{\BD}{\mathbf{D}}    \newcommand{\BE}{\mathbf{E}}    \newcommand{\BF}{\mathbf{F}}    \newcommand{\BG}{\mathbf{G}}    \newcommand{\BH}{\mathbf{H}}    \newcommand{\BI}{\mathbf{I}}    \newcommand{\BJ}{\mathbf{J}}    \newcommand{\BK}{\mathbf{K}}    \newcommand{\BL}{\mathbf{L}}
\newcommand{\BM}{\mathbf{M}}    \newcommand{\BN}{\mathbf{N}}    \newcommand{\BO}{\mathbf{O}}    \newcommand{\BP}{\mathbf{P}}    \newcommand{\BQ}{\mathbf{Q}}    \newcommand{\BR}{\mathbf{R}}    \newcommand{\BS}{\mathbf{S}}    \newcommand{\BT}{\mathbf{T}}    \newcommand{\BU}{\mathbf{U}}    \newcommand{\BV}{\mathbf{V}}    \newcommand{\BW}{\mathbf{W}}    \newcommand{\BX}{\mathbf{X}}
\newcommand{\BY}{\mathbf{Y}}    \newcommand{\BZ}{\mathbf{Z}}    

\newcommand{\bfsym}[1]{\ensuremath{\boldsymbol{#1}}}

 \def\balpha{\bfsym \alpha}
 \def\bbeta{\bfsym \beta}
 \def\bgamma{\bfsym \gamma}             \def\bGamma{\bfsym \Gamma}
 \def\bdelta{\bfsym {\delta}}           \def\bDelta {\bfsym {\Delta}}
 \def\bfeta{\bfsym {\eta}}              \def\bfEta {\bfsym {\Eta}}
 \def\bmu{\bfsym {\mu}}                 \def\bMu {\bfsym {\Mu}}
 \def\bnu{\bfsym {\nu}}
 \def\btheta{\bfsym {\theta}}           \def\bTheta {\bfsym {\Theta}}
 \def\beps{\bfsym \varepsilon}          \def\bepsilon{\bfsym \varepsilon}
 \def\bsigma{\bfsym \sigma}             \def\bSigma{\bfsym \Sigma}
 \def\blambda {\bfsym {\lambda}}        \def\bLambda {\bfsym {\Lambda}}
 \def\bomega {\bfsym {\omega}}          \def\bOmega {\bfsym {\Omega}}
 \def\brho   {\bfsym {\rho}}
 \def\btau{\bfsym {\tau}}
 \def\bxi{\bfsym {\xi}}
 \def\bzeta{\bfsym {\zeta}}
% May add more in future.
%%%%%%%%%%%%%%%%%%%%%%%%%%%%%%%%%%%%



\theoremstyle{plain}
\newtheorem{theorem}{\quad\quad Theorem}
\newtheorem{proposition}{\quad\quad Proposition}
\newtheorem{corollary}{\quad\quad Corollary}
\newtheorem{lemma}{\quad\quad Lemma}
\newtheorem{example}{Example}
\newtheorem{assumption}{\quad\quad Assumption}
\newtheorem{condition}{\quad\quad Condition}

\theoremstyle{definition}
\newtheorem{definition}{\quad\quad Definition}
\newtheorem{remark}{\quad\quad Remark}
\theoremstyle{remark}





\begin{document}
\title{
A Bayesian-motivated test for linear model in high-dimensional setting
}



\author[1]{Rui Wang}
\author[1,2]{Xingzhong Xu\thanks{Corresponding author\\Email address: xuxz@bit.edu.cn}}
\affil[1]{
School of Mathematics and Statistics, Beijing Institute of Technology, Beijing 
    100081,China
}
\affil[2]{
Beijing Key Laboratory on MCAACI, Beijing Institute of Technology, Beijing 100081,China
}

\maketitle
\begin{abstract}
    Using the idea of Bayesian factor, a new test for linear model in high-dimensional setting is proposed.

    Our theory is also useful in.
\end{abstract}
\section{Introduction} 
Consider linear regression model of the form
\begin{equation}\label{label:linearModel}
    \By = 
    \BX_a \bbeta_a + \BX_b \bbeta_b + \bepsilon, %\quad \bepsilon\sim \mathcal N_n(0,\phi^{-1} \BI_n),
\end{equation}
where $\By \in \mathbb R^n$ is the response, 
$\BX_a$, $\BX_b$ are $n\times q$ and $n\times p$ design matrices, respectively,  $\bbeta_a\in \mathbb R^q$, $\bbeta_b\in \mathbb R^p$ are unknown regression coefficients, and $\bepsilon=(\epsilon_1,\ldots,\epsilon_n)^\top$ are the iid errors with mean $0$ and covariance $\sigma^2=\phi^{-1}$.
Here we break the predictors into two parts $\BX_a$ and $\BX_b$ such that $\BX_a$ contains the predictors that are known to have effect on the response,
and we would like to test if $\BX_b$ contains any useful predictors.
That is, we are interested in testing the hypotheses
\begin{align}\label{theHypothesis}
    \mathcal H_0:   \bbeta_b =0,\quad
    \text{v.s.} \quad
    \mathcal H_1:   \bbeta_b \neq 0.
\end{align}
%Motivated by many recent applications of high dimensional regression, we consider the situation where $p+q$ is much larger than $n$.
The conventional test for hypotheses \eqref{theHypothesis} is the $F$-test which is also the likelihood ratio test under normal errors.
However, the $F$-test is not well defined in high dimensional setting.
In fact, if $\bepsilon$ is normal distributed and $\myRank[\BX_a;\BX_b]=n$, then the likelihood is unbounded under the alternative hypothesis.
This calls for new test methodologies in high-dimensional setting.

Two different high-dimensional settings have been extensively considered in the literature.
One is the small $p$, large $q$ setting.
An important example of this setting is testing individual coefficients of a high-dimensional regression.
See \cite{buhlmann2013statistical}, \cite{Zhang2013} and \cite{Lan2016} for testing procedures in this setting.
In this paper, however, we focus on the other setting, namely the large $p$, small $q$ setting.
In this case, there are just a few covariates, namely $\BX_a$, are known to have effect on the response, while there remain a large number of covariates, namely $\BX_b$, to be tested.
%We assume $\BX_a$ has full column rank and $\BX_b$ has full row rank.
In practice, which covariates belong to the part $\BX_a$ is determined apriori.
If no prior knowledge is available, $\BX_a$ can be $\mathbf 1_n$.
%In both cases, $q$ is often relatively small.
%However, there may remain a very large number of variables in $\BX_b$, so $p>n$.

Many test procedures have been proposed in the large $p$, small $q$ setting.
Based on an empirical Bayes model, \cite{Goeman2006} and \cite{Goeman2011} proposed a score test statistic as well as a method to determine the critical value of their test statistic.
This score test was further investigated by \cite{Lan2014Testing} and \cite{Lan2016a}.
Based on $U$-statistics,
\cite{Zhong2011Tests} proposed a test for the case where $\BX_a=\mathbf 1_n$.
Later, a generalization of this test to the general design matrix $\BX_a$ is proposed by \cite{Wang2015}.
To accomodate outlying observations and heavy-tailed distributions, \cite{Feng2013}
proposed a rank-based test for the entire coefficients.
\cite{Xu2016a} modified \cite{Feng2013}'s test and proposed a scalar invariant rank-based test.
Apart from the afore mentioned tests,
there is another line of research utilizing desparsified Lasso estimator; see \cite{zhang2016simultaneous} and the references therein.

Except for the test in \cite{Goeman2006}, most existing high dimensional tests adopted the random design assumption, that is, the rows of $\BX_b$ are considered as being generated from a super population.
As noted by \cite{Lei2018}, assuming a fixed design or a random design could lead to qualitatively different inferential results and the former is preferable from a theoretical point of view.
Hence in this paper, we focus on the fixed design setting.
Of course, our results are still valid for random design from a conditional inference perspective.




In Bayesian literature, many Bayesian tests have been proposed for hypothesis \eqref{theHypothesis} in low-dimensional setting; see \cite{javier2006Obj,Goddard2016,zhou2018On} and the references therein.
Although most existing Bayesian tests are not applicable in large $p$, small $q$ setting,
Bayes factor based tests are in principle suitable for high-dimensional testing problems for at least two reasons.
First, the Bayes factors corresponding to proper priors are always well defined, even if the likelihood is unbounded.
Second, under mild conditions, tests based on Bayes factors are admissible (See, e.g., \cite{Lehmann}, Theorem 6.7.2).
In fact, the test procedure in \cite{Goeman2006} is motivated by Bayesian methods but is treated as a frequentist significance test.


In this paper, we propose a new test statistic in large $p$, small $q$ setting which is the limit of Bayes factors under normal linear model.
We prove that, under mild conditions, the distribution of the proposed test statistic can be accurately approximated using Lindeberg's replacement trick.
And the critical value is determined by this approximated distribution.
Under normal error assumption, we also derive the asymptotic power funcion of the proposed test and the test in \cite{Goeman2006}.
A simulation is conducted to examine the performance of the proposed test.

The rest of the paper is organized as follows.




\section{Methodology}\label{sec:methodology}
Testing hypotheses \eqref{theHypothesis} in large $p$, small $q$ setting is a challenging problem.
%Based on an empirical Bayes model,
%\cite{Goeman2006} and \cite{Goeman2011} proposed a score test which can be used.
As \cite{Goeman2006} noticed, if $\bbeta_b \neq 0$ but $\BX_b \bbeta_b =0$, no test has any power.
They also pointed out that their test has negligible power for many alternatives and consequently is not unbiased.
For low-dimensional testing problems, a biased test is often regarded as problematic.
However, the following proposition shows that under normal assumption, there is no nontrivial unbiased test in large $p$, small $q$ setting.
\begin{proposition}\label{prop:unbiased}
    Suppose \eqref{label:linearModel} holds with
$\bepsilon\sim \mathcal N_n (0, \phi^{-1}\BI_n)$.
Suppose $\BX_a$ is an $n\times q$ matrix with full column rank, $q<n$ and $\myRank( [\BX_a;\BX_b])=n$.
Let $\varphi(\By)$ be a test function of level $\alpha$, that is, a Borel measurable function satisfying $0\leq \phi(\By)\leq 1$ and $\myE [\phi(\By)]\leq \alpha$ under the null hypothesis.
If $\phi(\By)$ is unbiased, that is $\myE [\phi(\By)]\geq \alpha$ under the alternative hypothesis, then $\varphi(\By)\equiv\alpha$, a.s.\ $\lambda$, where $\lambda(\cdot)$ is the Lebesgue measure on $\mathbb R^n$.
\end{proposition}
The above proposition implies that it is impossible to find a test with reasonable power for all alternatives.
This motivates us to adopt Bayesian methods % are natural choices in this case.
to find a test with good average power behavior.
Within the Bayesian framework, Bayes factor is commonly used for comparing two models.
In our problem, suppose $\bepsilon \sim \mathcal N_n (0,\phi^{-1} \BI_n)$, the Bayes factor for hypotheses \eqref{theHypothesis} is
\begin{equation*}
    B_{10}= \frac {
        \int d\mathcal N_n(\BX_a\bbeta_a + \BX_b\bbeta_b, \phi^{-1}\BI_n) (\By) \pi_1(\bbeta_b,\bbeta_a,\phi)\, \mathrm d\bbeta_b \mathrm d\bbeta_a \mathrm d\phi
}{
    \int d\mathcal N_n(\BX_a\bbeta_a, \phi^{-1} \BI_n)(\By) \pi_0(\bbeta_a,\phi) \, \mathrm d\bbeta_a \mathrm d\phi
    },
\end{equation*}
where $d\mathcal N_n(\mu, \bSigma)(\By)$ is the density function of a $\mathcal N_n(\mu,\bSigma) $ random vector with respect to the Lebesgue measure on $\mathbb R^n$,  $\pi_0(\bbeta_a,\phi)$ and $\pi_1(\bbeta_b,\bbeta_a,\phi)$ are the prior densities under the null and alternative hypotheses, respectively.
If $B_{10}$ is large, the alternative hypothesis is prefered.
%There have been several extensions of $g$-priors to $p>n$ case: \cite{maruyama2011}, \cite{Shang2011}.
The behavior of a Bayes factor largely depends on the choice of priors.
In Bayesian literature, many priors have been considered for testing the coefficients of linear model.
Popular priors include $g$-priors \citep{Liang2008Mixtures} and intrinsic priors \citep{Casella2006Obj}.
Unfortunately, these priors are not well defined in large $p$, small $q$ setting.
%To obtain valid Bayes factor, we use simple priors.

Note that under the null hypothesis $\mathcal H_0$, the model is low-dimensional.
This allows us to impose the reference prior $\pi_0 (\bbeta_a,\phi)=c/\phi$, where $c$ is a constant.
Under $\mathcal H_1$,
write $\pi_1(\bbeta_b,\bbeta_a,\phi)=\pi_1(\bbeta_b|\bbeta_a,\phi) \pi_1(\bbeta_a,\phi)$.
For parameters $\bbeta_a $ and $\phi$, we consider the same prior as in $\mathcal H_0$, that is $\pi_1(\bbeta_a,\phi)=\pi_0(\bbeta_a,\phi)$.
For parameter $\bbeta_b$, however, imposing the improper reference prior would not produce valid marginal density of $\By$.
%That is, the minimal training sample size is $q + p +1$.
%So we cannot impose the reference prior under $ H_1$ provided $q + p  \geq n$.
To make the marginal density of $\By$ well defined,
we consider the simple normal prior $p_1(\bbeta_b|\bbeta_a, \phi) = d \mathcal N_{p} (0, \kappa^{-1} \phi^{-1} \BI_{p}) (\bbeta_b) $, where $\kappa>0$ is a hyperparameter.
That is, we put the following priors,
\begin{equation}\label{priors}
    \pi_0(\bbeta_a,\phi)= \frac{c}{\phi},\quad
    \pi_1(\bbeta_b,\bbeta_a,\phi)=\frac{c}{\phi} d \mathcal N_{p} \left(0, \frac{1}{\kappa \phi} \BI_{p}\right) (\bbeta_b).
\end{equation}
%There are extansive literature consider the choice of $\kappa$.
%\cite{Kass1995} choose $\kappa$ such that the amount of information about the parameter equal to the amount of information contained in one observation.
%\begin{equation*}
    %\begin{split}
    %m_0(\By;\kappa, \tau) 
    %&:=
    %\int f_0^\tau (\By|\bbeta_a,\phi) \pi_0 (\bbeta_a, \phi) d\bbeta_a d\phi
    %\\
    %&=
    %\frac{
        %c_0 \Gamma\left( \frac{\tau n - q}{2}\right)
    %}{
        %\pi^{\frac{\tau n - q }{2}}
        %\tau^{\frac{\tau n}{2}}
        %|\BX_a^\top \BX_a|^{\frac{1}{2}}
        %\| (\BI_n -\BP_a) \By\|^{\tau n -q}
    %}.
    %\end{split}
%\end{equation*}
%\begin{equation*}
    %\begin{split}
        %m_1(\By;\kappa, \tau) 
    %&:=
    %\int f_1^\tau (\By|\bbeta_b,\bbeta_a,\phi) \pi_1 (\bbeta_b |\bbeta_a, \phi) \pi_1 (\bbeta_a, \phi)  d\bbeta_a d\bbeta_b d\phi
    %\\
    %&=
    %\frac{c_1\kappa^{\frac p 2} \Gamma \left(\frac{\tau n -q}{2}\right)}{
        %\pi^{\frac{\tau n -q}{ 2 }} \tau^{\frac{\tau n + p}{2}}
        %|\BX_a^\top \BX_a|^{\frac 1 2}
        %|\BX_b^{*\top} \BX_b^* + \frac{\kappa}{\tau } \BI_p|^{\frac 1 2}
    %}
    %\frac{1}{\left[ \By^{*\top} \By^* - \By^{*\top} \BX_b^* ( \BX_b^{*\top}\BX_b^* + \frac{\kappa}{\tau} \BI_p )^{-1} \BX_b^{*\top} \By^* \right]^{\frac{\tau n - q}{2}}}
    %.
    %\end{split}
%\end{equation*}
%\begin{equation*}
    %\begin{split}
        %\frac{m_1(\By;\kappa,\tau)}{m_0(\By;\kappa,\tau)} 
        %=
        %\frac{c_1 \kappa^{\frac{p}{2}}}{c_0 \tau^{\frac p 2}
        %|\BX_b^{*\top} \BX_b^* + \frac{\kappa}{\tau } \BI_p|^{\frac 1 2}
        %}
        %\left(
            %\frac{\By^{*\top} \By^*}{
%\By^{*\top} \By^* - \By^{*\top} \BX_b^* ( \BX_b^{*\top}\BX_b^* + \frac{\kappa}{\tau} \BI_p )^{-1} \BX_b^{*\top} \By^*
            %}
        %\right)^{\frac{\tau n - q }{2}}
    %\end{split}
%\end{equation*}

In what follows, we assume $\myRank (\BX_a )=q$ and $\myRank ([\BX_a;\BX_b])=n$.
Let $\BP_a=\BX_a(\BX_a^\top \BX_a)^{-1} \BX^\top$ be the projection matrix onto the column space of $\BX_a$.
Let $B_{10,\kappa} $ be the Bayes factor corresponding to the priors \eqref{priors}.
 It is straightforward to show that
%\begin{equation*}
    %\begin{split}
        %B_{10,\kappa}=  &
    %\frac{\kappa^{p/2}}{
        %|\BX_b^\top (\BI_n -\BP_a) \BX_b + \kappa \BI_p |^{1/2}
    %}
    %\cdot
    %\\
    %&
    %\left(
        %\frac{
            %\By\top (\BI_n-\BP_a) \By
        %}{
            %\By\top (\BI_n-\BP_a) \By
            %-
            %\By\top (\BI_n-\BP_a) \BX_b
            %\left(\BX_b^\top  (\BI-\BP_a) \BX_b + \kappa \BI_p\right)^{-1}
            %\BX_b^\top (\BI_n-\BP_a) \By
        %}
    %\right)^{(n-q)/2}.
    %\end{split}
%\end{equation*}
\begin{equation*}
    \begin{split}
        2\log (B_{10,{\kappa}}) =  &
    p\log \kappa
    -
        \log |\BX_b^\top (\BI_n -\BP_a) \BX_b + \kappa \BI_p |
    \\
    &
    -(n-q)\log \left(
            1-
        \frac{
            \By\top (\BI_n-\BP_a) \BX_b
            \left(\BX_b^\top  (\BI-\BP_a) \BX_b + \kappa \BI_p\right)^{-1}
            \BX_b^\top (\BI_n-\BP_a) \By
        }{
            \By\top (\BI_n-\BP_a) \By
        }
    \right).
    \end{split}
\end{equation*}

Denote by $\BI_n-\BP_a=\tilde{\BU}_a \tilde{\BU}_a^\top$ the rank decomposition of $\BI_n - \BP_a$, where $\tilde{\BU}_a$ is a $n\times (n-q)$ column orthogonal matrix.
Let $\BX_b^* = \tilde{\BU}_a^\top \BX_b$, $\By^* =\tilde \BU_a^\top \By$.
Let $\gamma_i$ be the $i$th largest eigenvalue of $\BX^*_b \BX_b^{*\top}$, $i=1,\ldots, n-q$.
Denote by $\BX_b^* =\BU_{b}^* \BD_{b}^* \BV_b^{*\top}$ the singular value decomposition of $\BX_{b}^*$, where  $\BU_{b}^*$, $\BV_b^*$ are $(n-q)\times (n-q)$ and $p\times (n-q)$ column orthogonal matrices, respectively, and $\BD_{b}^*=\mydiag (\sqrt {\gamma_1},\ldots, \sqrt{\gamma_{n-q}})$.
Then we have
\begin{equation*}
    \begin{split}
        2\log (B_{10,\kappa})
        %=&
         %p\log \kappa
         %- \sum_{i=1}^{n-q}\log ( \gamma_i + \kappa )
         %-(p-(n-q))\log \kappa
         %\\
         %&-(n-q)\log\left(1-\frac{\By^{*\top} \BX_b^* \left( \BX_b^{*\top} \BX_b^* + \kappa \BI_p \right)^{-1} \BX_b^{*\top} \By^* }{\By^{*\top} \By^*}\right)
         %\\
        %=&
         %- \sum_{i=1}^{n-q}\log ( \gamma_i + \kappa )
         %+(n-q)\log\left(\frac{\By^{*\top} \By^*}{\By^{*\top} \BU_b^*  \left[\frac 1 \kappa \left(\BI_{n-q}-\BD_b^{*} \left(  \BD_b^{*2} + \kappa \BI_{n-q} \right)^{-1} \BD_b^{*} \right) \right] \BU_b^{*\top} \By^* }\right)
         %\\
        =&
        (n-q)\log \kappa - \sum_{i=1}^{n-q}\log ( \gamma_i + \kappa )
         -(n-q)\log\left(1-\frac{\By^{*\top} \BU_b^*  \BD_b^{*} \left(  \BD_b^{*2} + \kappa \BI_{n-q} \right)^{-1} \BD_b^{*}   \BU_b^{*\top} \By^* }{\By^{*\top} \By^*}\right)
         .
    \end{split}
\end{equation*}
The main part of the above expression is
\begin{equation*}
    T_{\kappa} = \frac{\By^{*\top} \BU_b^*  \BD_b^{*} \left(  \BD_b^{*2} + \kappa \BI_{n-q} \right)^{-1} \BD_b^{*}   \BU_b^{*\top} \By^* }{\By^{*\top} \By^*}.
\end{equation*}
A large value of $T_{\kappa}$ supports the alternative hypothesis.
Hence $T_{\kappa}$ can be regarded as a frequentist test statistic.
The remaining problem is to choose an appropriate hyperparameter $\kappa$.
%Note that $\kappa$ controls the prior magnitude of $\bbeta_b$.
As $\kappa$ increases, the prior magnitude of $\bbeta_b$ decreases.
As \cite{Goeman2006} noted, the priors should place most probability on the alternatives which are perceived as more interesting to detect.
Their stretagy is to let the prior magnitude tend to zero to obtain a test with good power behavior under local alternatives, that is, $\|\bbeta_b\|$ is small.
In fact, if we let $\kappa$ tends to infinity, the limit
\begin{equation*}
    \lim_{\kappa\to \infty} \kappa T_{\kappa} = \frac{\By^{*\top} \BX_b^* \BX_b^{*\top} \By^* }{\By^{*\top} \By^*}
\end{equation*}
is exactly the test statistic of \cite{Goeman2006}.

As implied by Proposition \ref{prop:unbiased}, however, testing hypotheses \eqref{theHypothesis} in large $p$, small $q$ setting is a difficult problem and there is no nontrivial unbiased test.
%In fact, we shall see that for certain directions of $\bbeta_b$, the test in \cite{Goeman2006} has negligible power for any magnitude of $\|\bbeta_b\|$.
Hence it may be too ambitious to consider local power behavior provided the test is biased.
%In fact, the test of \cite{Goeman2006} may have negligible power even when $\|\bbeta_b\|$ is large.
%Hence it is questionable if a test with good local power behavior is a good choice in practice.
Thus, contrary to the strategy of \cite{Goeman2006}, we let $\kappa$ tend to $0$ to obtain a test with good power behavior for large $\|\bbeta_b\|$.
While the statistic $ T_{\kappa}$ itself degenerates to $1$ as $\kappa\to 0$,
the right derivative of $T_{\kappa}$ at $\kappa=0$ is well defined.
Thus, we proposed the following test statistic
%This motivates us to consider $B_{10}^0=\lim_{\kappa\to 0} B_{10}^\kappa$.
\begin{equation*}
    T=
    \left.\frac{\mathrm d T_{n,\kappa}}{\mathrm d \kappa}\right|_{\kappa=0}
    =
    -
     \frac{\By^{*\top} ( \BX_b^* \BX_b^{*\top} )^{-1} \By^* }{\By^{*\top} \By^*} .
\end{equation*}
The null hypothesis will be rejected if $T$ is large.




To formulate a valid frequentist test, we need to determine the critical value of $T$.
If $\bepsilon$ were indeed normally distributed, then under the null hypohtesis,
    $T \sim
    -
    {(\sum_{i=1}^{n-q} \gamma_i^{-1} z_i^2)}/{(\sum_{i=1}^{n-q} z_i^2)}$,
where $z_1,\ldots, z_{n-q}$ are iid $\mathcal N(0,1)$ random variables.
In this case, the exact critical value can be easily obtained.
However, normal distribution rarely appears in practice.
We would like to derive an asymptotic valid critical value for $T_{n}$ for general distributions of $\bepsilon$.

Under the null hypothesis,
\begin{equation*}
    T=-\frac{ (\sqrt \phi \bepsilon)^\top \tilde \BU_a (\BX_b^* \BX_b^{*\top})^{-1} \tilde \BU_a^\top (\sqrt \phi \bepsilon)}{ (\sqrt \phi \bepsilon)^\top \tilde \BU_a  \tilde \BU_a^\top (\sqrt \phi \bepsilon)}.
\end{equation*}
The numerator and the denominator of the above expression are both quadratic forms of iid random variables with mean $0$ and variance $1$.
Hence the key step towards the goal is to approximate the distribution of the quadratic form of iid standardized random variables.
The asymptotics of quadratic form have been extensively studied; see, e.g., \cite{jiang1996reml,Bentkus1996Optimal,Goetze2002,Dicker2015Flexible,Bai2017}.
Most existing work use normal distribution to approximate the distribution of the quadratic form.
However, normal distribution is just one of the possible limit distributions of quadratic form.
See \cite{Sevast1961A} for a full characterazation of the limit distributions of quadratic form of normal random variables.
Our approximation stretegy is to replace the random variables in quadratic form by suitable normal random variables.
The error bound of this approximation will be derived by Lindeberg's replacement trick (see, e.g., \cite{pollard1984convergence}, Section III.4).

Let $\mathscr C^4(\mathbb R)$ denote the class of all bounded real functions on $\mathbb R$ having bounded, continuous $k$th derivatives, $1\leq k\leq 4$.
It is known that if $\myE f(Z_n)\to \myE f(Z)$ for every $f\in \mathscr C^4 (\mathbb R)$ then $Z_n \rightsquigarrow Z$; see, e.g., \cite{pollard1984convergence}, Theorem 12 of Chapter III.
We have the following approximation theorem.


\begin{theorem}\label{TheoremLindeberg}

    Let $\bxi=(\xi_1,\ldots,\xi_n)^\top$, where $\xi_i$'s are iid random variable with $\myE \xi_1=0$, $\myVar (\xi_1)=1$.
    Furthermore, suppose the distribution of  $\xi_1$ is symmetric about $0$ and has finite eighth moments.
    Let $\BA$ be an $n\times n$ symmetric matrix with elements $a_{i,j}$.
Define
    \begin{equation*}
        S=\frac{
            \bxi^\top \BA \bxi-\mytr (\BA)
        }{
            \sqrt{
    2 \mytr(\BA^2)
    +
    (\myE (\xi_1^4)-3)\sum_{i=1}^n a_{i,i}^2
            }             
        }.
    \end{equation*}
    Let $z_1,\ldots,z_n$  be iid random variables with distribution $\mathcal N (0, 1) $ and $\check z_1, \ldots, \check z_n$ be iid random variables with distribution $\mathcal N (0,1)$ which are independent of $\xi_1,\ldots, \xi_n$.
    Let $\tau$ be a real number.
    Define
    \begin{equation*}
        S_\tau^* =
        \frac{
            \tau \sum_{i=1}^n  a_{i,i}\check z_i
        +2\sum_{1\leq i <j \leq n} a_{i,j} z_i z_j
    }
    {
            \sqrt{
    2 \mytr(\BA^2)
    +
    (\myE (\xi_1^4)-3) \sum_{i=1}^n a_{i,i}^2
            }             
        }.
    \end{equation*}
    Then for every $f\in \mathscr C^4(\mathbb R)$,
    \begin{equation}\label{eq:longInequality}
        \begin{split}
             &
              \left| \myE f(S)-\myE f(S_\tau^*)\right|
             \\
\leq&
\frac{
\left|
\myE (\xi_1^4)-1
            -
            \tau^2
\right|
}{2}
\|f^{(2)}\|_\infty
\frac{
    \sum_{i=1}^n a_{i,i}^2
}{
    2 \mytr(\BA^2)
    +
    (\myE (\xi_1^4)-3) \sum_{i=1}^n a_{i,i}^2
}
\\
&
            +
            \frac{
            \max\left(
    \left|\myE (\xi_1^2-1)^3\right|
            ,
12 (\myE (\xi_1^4)-1)
        \right)
            }{6} \|f^{(3)}\|_\infty
            \frac{
            \sum_{l=1}^n 
            \left(|a_{l,l}|
         \sum_{i=1}^{n} a_{i,l}^2 
     \right)
 }{
    \left(
        2 \mytr(\BA^2)
    +
    (\myE (\xi_1^4)-3) \sum_{i=1}^n a_{i,i}^2
\right)^{3/2}
 }
         \\
            &+
            \frac{
             16 \myE (\xi_1^8) + 80 \myE (\xi_1^4) + 3\tau^4 + 96 
            }{24} \|f^{(4)} \|_{\infty} 
            \frac{
                \sum_{l=1}^n \left( \sum_{i=1}^n a_{i,l}^2 \right)^2
            }{
            \left(
        2 \mytr(\BA^2)
    +
    (\myE (\xi_1^4)-3)\sum_{i=1}^n a_{i,i}^2
\right)^{2}
            }
            .
        \end{split}
    \end{equation}
    \label{approximation}
\end{theorem}
\begin{remark}\label{remark1}
    If $\tau^2=\myE (\xi_1^4) -1$, the first term of the right hand side of \eqref{eq:longInequality} disappear.
    In practice, however, the quantity $\myE (\xi_1^4)$ is often unknown and
    $\tau^2$ should be chosen as an estimator of $\myE (\xi_1^4)-1$.
\end{remark}

\begin{remark}
%Let $\rho$ denote any metric on the set of probability measures $\mathcal M (\mathbb R)$ on $\mathbb R$ such that convergence in $(\mathcal M (\mathbb R), \rho)$ is equivalent to weak convergence.
    As noted in \cite{Chatterjee2008}, Section 3.1, an almost necessary condition for the asymptotic normality of $S$ is
    \begin{equation}\label{eq:normalC}
        \frac{\mytr (\BA^4)}{
            \left( 
        2 \mytr(\BA^2)
    +
    (\myE (\xi_1^4)-3)\sum_{i=1}^n a_{i,i}^2
            \right)^2
        }\to 0. 
    \end{equation}
    On the other hand, it can be seen that the right hand side of \eqref{eq:longInequality} converges to $0$ provided $\tau^2=\myE (\xi_1^4)-1$ and
    \begin{equation}\label{eq:generalC}
            \frac{
                \sum_{l=1}^n \left( \sum_{i=1}^n a_{i,l}^2 \right)^2
            }{
            \left(
        2 \mytr(\BA^2)
    +
    (\myE (\xi_1^4)-3)\sum_{i=1}^n a_{i,i}^2
\right)^{2}
            }
            \to 0.
    \end{equation}
    It can be seen that \eqref{eq:generalC} is much weaker than \eqref{eq:normalC}.
    For example, if $a_{i,j}=1$, $i=1,\ldots, n$, $j=1,\ldots, n$ and $\myE (\xi_1^4)=3$, then the condition \eqref{eq:generalC} holds but the condition \eqref{eq:normalC} does not hold.
\end{remark}


We now apply Theorem \ref{approximation} to approximate the null distribution of the proposed statistic $T$.
Note that under the null hypothesis,
\begin{equation*}
    T+\frac{\mytr\left( (\BX_b^* \BX_{b}^{*\top})^{-1} \right)}{n-q} 
    =\frac{ (\sqrt \phi \bepsilon)^\top 
        \left( 
    -\tilde \BU_a (\BX_b^* \BX_b^{*\top})^{-1} \tilde \BU_a^\top
 + \frac{\mytr\left( (\BX_b^* \BX_b^{*\top})^{-1} \right)}{n-q}\tilde{\BU}_b \tilde{\BU}_b^\top
        \right)
    (\sqrt \phi \bepsilon)}{ (\sqrt \phi \bepsilon)^\top \tilde \BU_a  \tilde \BU_a^\top (\sqrt \phi \bepsilon)},
\end{equation*}
where the numerator has zero mean and the denominator is close to $n-q$.
In what follows, let $\bxi=\sqrt \phi \bepsilon$ and 
    \begin{equation*}
        \BA= -\tilde{\BU}_a (\BX_{b}^* \BX_b^{*\top})^{-1} \tilde{\BU}_a^\top 
        + \frac{\mytr\left( (\BX_b^* \BX_b^{*\top})^{-1} \right)}{n-q}\tilde{\BU}_a \tilde{\BU}_a^\top.
    \end{equation*}
    Let $F_{\tau} (x)$ be the cumulative distribution function of $
            \tau \sum_{i=1}^n  a_{i,i}\check z_i
        +2\sum_{1\leq i <j \leq n} a_{i,j} z_i z_j
        $.
        As noted in Remark \ref{remark1}, $\tau^2$ should be a consistent estimator of $\phi^2 \myE (\epsilon_1^4)-1$ under the null hypothesis.

\begin{theorem}
    Suppose the conditions of Theorem \ref{TheoremLindeberg} hold. Furthermore, suppose as $n\to \infty$, the condition \eqref{eq:generalC} holds.
        Let $\hat \tau^2$ be an consistent estimator of $\phi^2 \myE (\epsilon_1^4)-1$ based on $\BX$, $\By$.
    Then
    \begin{equation*}
            \Pr\left( T > \frac{F^{-1}_{\hat \tau} (1-\alpha)-\mytr\left( (\BX_b^* \BX_b^{*\top})^{-1}  \right)}{n-q} \right)
            \to \alpha.
    \end{equation*}
    \label{thm:criticalValue}
\end{theorem}

        A consistent estimator of $\sigma^{-4} \myE (\epsilon_1^4)-1$ has already appeared in \cite{Bai2017} based on the standardized residuals.
        Here we use a slightly different estimator which is based on the ordinary least squares residuals $\tilde \bepsilon=(\tilde \epsilon_1,\ldots, \tilde \epsilon_n)^\top=(\BI_n - \BP_a) \By$.
    From \cite{Bai2017}, Theorem 2.1, 
    \begin{align*}
        &\myE \left( \tilde \bepsilon^\top \left( \BI_n-\BP_a \right) \tilde \bepsilon \right)
        = (n-q) \sigma^2,
        \\
        &\myE \left( \sum_{i=1}^n \tilde \epsilon_i^4 \right)
        =
        3\sigma^4 \mytr (\BI_n-\BP_a)^{\circ 2} 
        +\left( \myE (\epsilon_1^4)-3 \sigma^{4}\right)
        \mytr \left( (\BI_n-\BP_a)^{\circ 2}  \right)^2.
    \end{align*}
    Then a moment estimator of $\sigma^{-4}\myE (\epsilon_1^4)-1$ is 
\begin{equation*}
    \hat \tau^2 =
    \frac{
        \displaystyle\frac{(n-q)^2 \sum_{i=1}^n \tilde{\epsilon}_i^4}{\left( \tilde \bepsilon^\top \left( \BI_n - \BP_a \right) \tilde \bepsilon \right)^2}
        - 3\mytr (\BI_n-\BP_a)^{\circ 2}
    }{
        \mytr\left( \left( \BI_n-\BP_a \right)^{\circ 2} \right)^2
    }
    +2.
\end{equation*}

\begin{proposition}\label{prop:estimation}
    Suppose the conditions of \ref{TheoremLindeberg} holds for $\bxi=\sqrt \phi \bepsilon$.
    Suppose $q/n\to 0$.
    Then under the null hypothesis, $\hat \tau^2 \xrightarrow{P} \sigma^{-4} \myE (\epsilon_1^4)-1$.
\end{proposition}
We reject the null hypothesis if 
\begin{equation*}
    T > \frac{F^{-1}_{\hat \tau} (1-\alpha)-\mytr\left( (\BX_b^* \BX_b^{*\top})^{-1}  \right)}{n-q}.
\end{equation*}
This test procedure is asymptotically exact of size $\alpha$ under the conditions of Theorem \ref{thm:criticalValue} and Proposition \ref{prop:estimation}.


\section{Power analysis}
In this section, we investigate the asymptotic power of the proposed test procedure as well as the test in \cite{Goeman2006}.
To make the expression of the asymptotic power functions tractable, we shall assume further conditions so that the test statistics are asymptotically normally distributed.
Also, $\bepsilon$ is assumed to be normally distributed so that we can can obtain the full power function rather than only local power function.

To derive the asymptotic power of the proposed test and the test in \cite{Goeman2006} simultaneously, we consider the general statistic
    ${
\By^{*\top}
    (\BX_b^* \BX_b^{*\top})^{k} 
        \By^*
    }/{\By^{*\top} \By^*}$.
    In fact, the proposed test statistic corresponds to $k=-1$ while the test statistic in \cite{Goeman2006} corresponds to $k=1$.
    Note that for any $x\in \mathbb R$,
    \begin{equation*}
        \Pr\left( 
    \frac{
\By^{*\top}
    (\BX_b^* \BX_b^{*\top})^{k} 
        \By^*
    }{\By^{*\top} \By^*}
    \leq x
        \right)
        =
        \Pr\left( 
\By^{*\top}
\left( 
    (\BX_b^* \BX_b^{*\top})^{k} 
    -x\BI_{n-q}
\right)
        \By^*
    \leq 
    0
\right).
    \end{equation*}
Hence the asymptotic behavior of noncentral quadratic form will play a key role in our investigation.
We have the following proposition.
\begin{proposition}
    Let $Z=(z_1,\ldots, z_n)^\top$, where $z_i$'s are iid $\mathcal N(0,1)$ random variables.
    Let $\BA$ be an $n\times n$ symmetric matrix with elements $a_{i,j}$.
    Let $\Bb=(b_1,\ldots, b_n)^\top$ be an $n$ dimensional vector.
    If $\mytr(\BA^4)/\mytr^2(\BA^2)\to 0$,
    then
    \begin{equation*}
        \frac{Z^\top \BA Z + b^\top Z - \mytr(\BA)}{\sqrt{2\mytr(\BA^2)+\|\Bb\|^2}}
        \rightsquigarrow \mathcal N(0,1).
    \end{equation*}
    \label{Lemma:normal}
\end{proposition}
\begin{remark}
    Proposition \ref{Lemma:normal} does not impose any condition on $\Bb$.
    This allows us to give the full asymptotic power function of tests.
    As the cost of this flexibility, we have to make the normal assumption.
\end{remark}


Now we investigate the asymptotic behavior of
    ${
\By^{*\top}
    (\BX_b^* \BX_b^{*\top})^{k} 
        \By^*
    }/{\By^{*\top} \By^*}$.
Let $w_i=(\BV_b^{*\top} \bbeta_b)_i$ be the coordinate of $\beta_b$ along the $i$th principal component direction of $\BX_b^{*\top} \BX_b^*$, $i=1,\ldots, n-q$.
It turns out that many quantities involved can be conveniently represented as the expectations with respect to $I$, a random variable uniformly distributed on $\{1,\ldots, n-q\}$.
For example, $
        {
            \mytr(\BX_b^* \BX_b^{*\top})^k
        }/
        (n-q )
        =\myE (\gamma_I^k)
$.

\begin{theorem}\label{generalTheorem}
    Suppose model \eqref{label:linearModel} holds with $\bepsilon\sim \mathcal N_n (0,\phi^{-1} \BI_n)$.
    Let $k\neq 0$ be a fixed number.
    Suppose as $n\to \infty$, $n-q\to \infty$ and
\begin{equation}
    \frac{
        \max_{1\leq i \leq n-q}
        \left( 
        \gamma_i^k
            -
                \myE (\gamma_I^k)
        \right)^2
    }{
        (n-q) \myVar (\gamma_I^k)
    }\to 0.
    \label{eq:toBeCondition}
\end{equation}
Then for any $x\in \mathbb R$,
\begin{equation*}
    \begin{split}
    &\Pr\left( 
        \frac{
            \By^{*\top} \left( \BX_b^* \BX_b^{*\top} \right)^k \By^*
        }{
            \By^{*\top} \By^*
        } 
        \leq 
        \myE (\gamma_I^k)
        +\sqrt{
            \frac{2\myVar\left( \gamma_I^k \right)}{n-q} 
        }
        x
    \right) 
    \\
    =&
    \Phi\left( 
        \frac{
            \left( \myE (\gamma_I w_I^2) + \phi^{-1} \right)
            \sqrt{{2\myVar\left( \gamma_I^k \right)}} 
            x
            -
            \sqrt{n-q}
            \myCov\left( \gamma_I^k, \gamma_I w_I^2 \right)
        }{
            \sqrt{
                2\phi^{-2} \myVar ( \gamma_I^k ) 
                +
                4\phi^{-1}
     \myE\left[ 
        \left( \gamma_I^k -\myE(\gamma_I^k) -\sqrt{\frac{2\myVar (\gamma_I^k)}{n-q}}x \right)^2
        \gamma_I w_I^2
    \right]
            }
        } 
    \right)
    +o(1)
    .
    \end{split}
\end{equation*}

\end{theorem}
Under the conditions of Theorem \ref{generalTheorem}, the proposed test should reject the null hypothesis when
\begin{equation*}
        \frac{
            \By^{*\top} \left( \BX_b^* \BX_b^{*\top} \right)^{-1} \By^*
        }{
            \By^{*\top} \By^*
        } 
        \leq 
        \myE (\gamma_I^{-1})
        +\sqrt{
            \frac{2\myVar\left( \gamma_I^{-1} \right)}{n-q} 
        }
        \Phi^{-1}(\alpha),
\end{equation*}
and the asymptotic power function of the proposed test is
\begin{equation}\label{eq:powerProposed}
    \begin{split}
    \Phi\left( 
        \frac{
            \left( \myE (\gamma_I w_I^2) + \phi^{-1} \right)
            \sqrt{2\myVar\left( \gamma_I^{-1} \right)} 
            \Phi^{-1}(\alpha)
            +
            \sqrt{n-q}
            \myCov\left( - \gamma_I^{-1}, \gamma_I w_I^2 \right)
        }{
            \sqrt{
                2\phi^{-2} \myVar ( \gamma_I^{-1} ) 
                +
                4\phi^{-1}
        \myE\left[ 
            \left( \gamma_I^{-1} -\myE(\gamma_I^{-1}) -\sqrt{\frac{2\myVar (\gamma_I^{-1})}{n-q}} \Phi^{-1}(\alpha) \right)^2
        \gamma_I w_I^2
    \right]
            }
        } 
    \right).
    \end{split}
\end{equation}
On the other hand, the test in \cite{Goeman2006} should reject the null hypothesis when
\begin{equation*}
        \frac{
            \By^{*\top} \BX_b^* \BX_b^{*\top} \By^*
        }{
            \By^{*\top} \By^*
        } 
        >
        \myE (\gamma_I)
        +\sqrt{
            \frac{2\myVar\left( \gamma_I \right)}{n-q} 
        }
        \Phi^{-1}(1-\alpha),
\end{equation*}
and the asymptotic power function of their test is
\begin{equation}\label{eq:powerTheirs}
    \begin{split}
    &\Phi\left( 
        \frac{
            \left( \myE (\gamma_I w_I^2) + \phi^{-1} \right)
            \sqrt{2\myVar\left( \gamma_I \right)} 
            \Phi^{-1}(\alpha)
            +
            \sqrt{n-q}
            \myCov\left( \gamma_I, \gamma_I w_I^2 \right)
        }{
            \sqrt{
                2\phi^{-2} \myVar ( \gamma_I ) 
                +
                4\phi^{-1}
    \myE\left[ 
        \left( \gamma_I -\myE(\gamma_I) +\sqrt{\frac{2\myVar (\gamma_I)}{n-q}} \Phi^{-1}(\alpha) \right)^2
        \gamma_I w_I^2
    \right]
            }
        } 
    \right).
    \end{split}
\end{equation}

It can be seen from \eqref{eq:powerProposed} and \eqref{eq:powerTheirs} that the power of the proposed test mainly depends on $\myCov (-\gamma_I^{-1}, \gamma_I w_I^2)$ while the power of the test in \cite{Goeman2006} mainly depends on $\myCov(\gamma_I, \gamma_I w_I^2)$.  
Unfortunately, neither of these two quantities is positive definite no matter how strong the signal $\myE (w_I^2)$ is.
This fact is not surprising in view of Proposition \ref{prop:unbiased}.

On the other hand, $\myCov (-\gamma_I^{-1}, \gamma_I w_I^2)$ and $\myCov(\gamma_I, \gamma_I w_I^2)$ are positive definite if the parameter $\bbeta_b$ is restricted  in certain subspaces of $\mathbb R^p$.
Let $d_1$ be the maximum $i$ such that $ \gamma_i^{-1} < \myE (\gamma_I^{-1})$.
Let $d_2$ be the maximum $i$ such that $\gamma_i > \myE (\gamma_I) $.
Then it can be seen that
\begin{align*}
\myCov (-\gamma_I^{-1}, \gamma_I w_I^2)
&=
\frac{1}{n-q} \sum_{i=1}^{n-q} \left(\myE (\gamma_I^{-1})-\gamma_i^{-1} \right) \gamma_i w_i^2
\end{align*}
is positive if $w_{d_1+1}=\cdots =w_{n-q}=0$, while
\begin{align*}
    \myCov(\gamma_I, \gamma_I w_I^2)
&=
\frac{1}{n-q} \sum_{i=1}^{n-q} \left(\gamma_I - \myE \gamma_i \right) \gamma_i w_i^2
\end{align*}
is positive if $w_{d_2 + 1}=\cdots = w_{n-q}=0$.
In other words, $\myCov(-\gamma_I^{-1}, \gamma_I w_I^2)$ is positive definite if $\bbeta_b$ belongs to the rank $d_1$ principal component subspace of $\BX_{b}^{*\top}\BX_b^{*}$, and $\myCov(\gamma_I, \gamma_I w_I^2)$ is positive definite if $\bbeta_b$ belongs to the rank $d_2$ principal component subspace of $\BX_{b}^{*\top}\BX_b^{*}$.
Note that if $\gamma_i > \myE (\gamma_I)$, then $\gamma_i^{-1} < (\myE (\gamma_I))^{-1}\leq \myE (\gamma_I^{-1})$, where the last inequality follows from Jensen's inequality.
Consequently, $ d_1 \geq d_2 $.
This implies that compared with the test in \cite{Goeman2006}, the proposed test can detect the signals from more directions.



\section{Numerical results}

In this section, we conduct simulations to examine the performance of the proposed test.





\section{Conclusions}


\section*{Acknowledgments}
This work was supported by the National Natural Science Foundation of China under Grant No.\ 11471035.




\begin{appendices}

    \section{Proofs of the results in Section \ref{sec:methodology}}

\begin{proof}[\textbf{Proof of Proposition \ref{prop:unbiased}}]
    Since $\myRank([\BX_a,\BX_b])=n$, $\phi(\By)$ is unbiased if and only if
    \begin{equation*}
        \int_{\mathbb R^n} \varphi(\By) d\mathcal N_n (\mu,\phi^{-1} \BI_n) (\By) \,\mathrm{d} \By\geq \alpha
        \quad \text{for all } \mu\in \mathbb R^n.
    \end{equation*}
    %where $\mathcal N_n(\mu,\sigma^2 \BI_n)(d\By)$ is the density function of a $\mathcal N_n (\mu, \sigma^2 \BI_n)$ random vector with respect to Lebesgue measure $\lambda(\cdot)$.
    From \cite{Lehmann}, Theorem 2.7.1, $\myE [\phi(\By)]=\alpha$ under the null hypothesis.
    In particular,
    we have
    \begin{equation}\label{eq:unbiasedproof1}
        \int_{\mathbb R^n} [\varphi(\By)-\alpha] d\mathcal N_n (0,\phi^{-1} \BI_n)(\By) \,\mathrm{d} \By=0.
    \end{equation}
    If $\alpha=0$ or $1$, the conclusion is trivially true.
    In what follows, we assume $0<\alpha<1$.
    We claim that if $\varphi(\By) \geq \alpha$, a.s.\ $\lambda$, then the conclusion holds.
    In fact, if $\varphi(\By) \geq \alpha$, a.s.\ $\lambda$,
    then the integrand of \eqref{eq:unbiasedproof1} is nonnegative, and hence must be $0$ a.s.\ $\lambda$, which implies the conclusion. 
    Next we prove $\varphi(\By) \geq \alpha$, a.s.\ $\lambda$ by contradiction. Suppose $\lambda(\{\By:\varphi (\By) <\alpha\})>0$.
    Then there exists a sufficiently small $\eta >0$, such that $\lambda(\{\By:\varphi (\By) <\alpha-\eta\})>0$.
    We denote $ E:=\{\By\in \mathbb R^n:\varphi (\By) <\alpha-\eta\}$.
    From Lebesgue density theorem \citep[Corollary 6.2.6]{book:992991}, there exists a point $z\in  E$, such that, for any $\epsilon >0$ there is a $\delta_{\epsilon}>0$ such that
    \begin{equation*}
        \left|\frac{\lambda(E^\complement\cap C_{\epsilon})}{\lambda(C_{\epsilon})}\right|<\epsilon,
    \end{equation*}
    where $C_{\epsilon}=\prod_{i=1}^n [z_i-\delta_{\epsilon}, z_i + \delta_{\epsilon}]$.
    We put
    \begin{equation*}
        \epsilon=\left(\frac{\sqrt \pi}{\sqrt 2 \Phi^{-1}\left(1-\frac{\eta}{6n}\right)}\right)^n \frac{\eta}{3},
    \end{equation*}
    where $\Phi(\cdot)$ is the cumulative distribution function of a standard normal random variable.
    Then for any $\phi>0$,
    \begin{equation*}
        \begin{split}
            \alpha \leq& 
            \int_{\mathbb R^n}\varphi(\By) d\mathcal N_n (z, \phi^{-1} \BI_n) (\By)\,\mathrm{d} \By
            \\
            =&
            \int_{E\cap C_{\epsilon}}\varphi(\By) d\mathcal N_n (z, \phi^{-1} \BI_n) (\By)\,\mathrm{d} \By
            +
            \int_{E^\complement\cap C_{\epsilon}}\varphi(\By) d\mathcal N_n (z, \phi^{-1} \BI_n) (\By)\,\mathrm{d} \By
            +
            \int_{C_{\epsilon}^\complement}\varphi(\By) d\mathcal N_n (z, \phi^{-1} \BI_n) (\By)\,\mathrm{d} \By
            \\
            \leq&
            \alpha-\eta
            +
            \int_{E^\complement\cap C_{\epsilon}} d\mathcal N_n (z, \phi^{-1} \BI_n) (\By)\,\mathrm{d} \By
            +
            \int_{C_{\epsilon}^\complement} d\mathcal N_n (z, \phi^{-1} \BI_n) (\By)\,\mathrm{d} \By
            \\
            \leq&
            \alpha-\eta
            +
            \left(\frac{\phi}{2\pi}\right)^{n/2}\lambda(E^\complement\cap C_{\epsilon})
            +
            2n\left(1-\Phi(\sqrt \phi \delta_\epsilon)\right)
            \\
            \leq&
            \alpha-\eta
            +
            \left(\frac{\phi}{2\pi}\right)^{n/2}
            \epsilon
            (2\delta_\epsilon)^n
            +
            2n\left(1-\Phi(\sqrt \phi \delta_\epsilon)\right)
            \\
            =&
            \alpha-\eta
            +
            \left(\frac{\sqrt{\phi} \delta_{\epsilon}}{\Phi^{-1}\left(1-\frac{\eta}{6n}\right)}\right)^{n}
            \frac{\eta}{3}
            +
            2n\left(1-\Phi(\sqrt \phi \delta_\epsilon)\right).
        \end{split}
    \end{equation*}
    Putting 
    \begin{equation*}
        \phi = \left(\frac{\Phi^{-1}\left(1-\frac{\eta}{6n}\right)}{\delta_\epsilon}\right)^2
    \end{equation*}
    yields the contradiction $\alpha\leq \alpha-(2/3)\eta$.
    This completes the proof.

\end{proof}





\section{haha3}
\begin{proof}[\textbf{Proof of Theorem \ref{TheoremLindeberg}}]
    Let
    \begin{equation*}
        \tilde a_{i,j}:=
        \frac{a_{i,j}}{
            \sqrt{
    2\mytr(\BA^2)
    +
    (\myE(\xi_1^4)-3)\mytr(\BA\circ \BA)
            }             
        }
        .
    \end{equation*}
Then
    \begin{equation*}
        S=\sum_{i=1}^n \tilde a_{i,i}(\xi_i^2-1)
        +2\sum_{1\leq i < j \leq n}  \tilde a_{i,j} \xi_i \xi_j,
\quad
        S_\tau^* =\tau \sum_{i=1}^n \tilde a_{i,i}\check z_i
        +2\sum_{1\leq i <j \leq n} \tilde a_{i,j} z_i z_j.
    \end{equation*}
    For $l=1,\ldots, n$, define
    \begin{align*}
        S_l = & 
        \sum_{i=1}^{l-1} \tilde a_{i,i}(\xi_i^2-1)
        +
        \tau\sum_{i=l+1}^{n} \tilde a_{i,i}  \check  z_i
        +2\sum_{1\leq i <j \leq l-1}  \tilde a_{i,j} \xi_i \xi_j
        +2\sum_{i=1}^{l-1} \sum_{j=l+1}^n \tilde a_{i,j} \xi_i z_j
        +2\sum_{l+1 \leq i < j \leq n}  \tilde a_{i,j} z_i z_j
        ,
        \\
        h_l = & \tilde a_{l,l} (\xi_l^2 -1)
        +2\sum_{i=1}^{l-1} \tilde a_{i,l} \xi_i \xi_l
        +2\sum_{i =l +1}^n \tilde a_{i,l} z_i \xi_l
        ,
        \\
        g_l = &
        \tau \tilde a_{l,l} \check z_l
        +2\sum_{i =1}^{l-1} \tilde a_{i,l} \xi_i z_l
        +2\sum_{i = l+1}^n \tilde a_{i,l} z_i z_l
        .
    \end{align*}
    It can be seen that for $l=2,\ldots, n$, 
    $S_{l-1}+ h_{l-1} =S_{l} + g_{l} $, and
    $S=S_n + h_n$, $S_1 + g_1=S_\tau^*$.
    %\begin{equation*}
        %S_n + h_n =  \sum_{i=1}^n \tilde a_{i,i}(\xi_i^2 - 1)
        %+2 \sum_{1\leq i < j \leq n} \tilde a_{i,j} \xi_i \xi_j,
    %\end{equation*}
    %\begin{equation*}
        %S_1 + g_1 = \sum_{i=1}^n \tilde a_{i,i}\check z_i
        %+2\sum_{1\leq i <j \leq n} \tilde a_{i,j} z_i z_j.
    %\end{equation*}

    Thus, for any $f \in \mathscr C^4 (\mathbb R)$,
    \begin{equation*}
        \begin{split}
        \left|\myE f\left(S\right)
        -
        \myE f\left(S_\tau^*\right)\right| 
        =&
        \left| \myE f(S_n+h_n)-\myE f(S_1+g_1)\right|
        \\
        =&
        \left|\sum_{l=2}^{n} \left(\myE f(S_{l}+h_{l})-\myE f(S_{l-1}+h_{l-1})\right)+\myE f(S_{1}+h_{1})-\myE f(S_{1}+g_{1})\right|
        \\
        = &
       \left| \sum_{l=1}^{n} \myE f(S_{l}+h_{l})-\myE f(S_{l}+g_{l})\right|
       .
        \end{split}
    \end{equation*}
    Apply Taylor's theorem, for $l=1,\ldots,n$,
    \begin{equation*}
        \begin{split}
            f(S_{l}+h_{l})=&
            f(S_{l})
            +
            \sum_{k=1}^3
            \frac{1}{k!} h_l^k f^{(k)} (S_{l})
            +
            \frac{1}{24}h_{l}^4 f^{(4)} (S_{l}+\theta_1 h_{l}),
            \\
            f(S_{l}+g_{l})=&
            f(S_{l})
            +
            \sum_{k=1}^3
            \frac{1}{k!} g_l^k f^{(k)} (S_{l})
            +
            \frac{1}{24}g_{l}^4 f^{(4)} (S_{l}+\theta_{2} g_{l}),
        \end{split}
    \end{equation*}
    where $\theta_1,\theta_2\in[0,1]$.
    Thus,
    \begin{equation*}
        \begin{split}
             &\left| \myE f(S_{l}+h_{l})-\myE f(S_{l}+g_{l})\right|
\leq
\left|
            \sum_{k=1}^3
            \frac{1}{k!} \myE f^{(k)} (S_{l})
            \myE_l (h_l^k - g_l^k)
            \right|
            +
            \frac{1}{24} \|f'''' \|_{\infty} \left(\myE (h_{l}^4)+\myE (g_{l}^4)\right),
        \end{split}
    \end{equation*}
where $\myE_l$ denotes taking expectation with respect to $\xi_l, z_l ,\check z_l$.
It is straightforward to show that
\begin{equation*}
    \begin{split}
        \myE_l (h_l-g_l)&=0, 
        \\
        \myE_l (h_l^2-g_l^2)&= 
        \left(
            \myE (\xi_1^4) - 1
            -
            \tau^2
        \right)
        \tilde a_{l,l}^2
        ,
        \\
        \myE_l (h_l^3-g_l^3)&= 
        \myE (\xi_1^2-1)^3
        \tilde a_{l,l}^3 
        +
        12 (\myE (\xi_1^4) - 1) \tilde a_{l,l} \left( \sum_{i=1}^{l-1} \tilde a_{i,l} \xi_i + \sum_{i=l+1}^n \tilde a_{i,l} z_i    \right)^2 
        .
    \end{split}
\end{equation*}
Thus,
\begin{equation}\label{eq:Lin1}
        \begin{split}
             &\left| \myE f(S_{l}+h_{l})-\myE f(S_{l}+g_{l})\right|
             \\
\leq&
\frac{1}{2}
\|f^{(2)}\|_\infty
\left|
\myE(\xi_1^4)-1
            -\tau^2
\right|
        \tilde a_{l,l}^2
        \\
            &+
            \frac{1}{6} \|f^{(3)}\|_{\infty}
\left(
    \left|\myE (\xi_1^2-1)^3\right|
        |\tilde a_{l,l}^3 |
        +
        12 (\myE (\xi_1^4)-1) |\tilde a_{l,l}|
            \myE 
            \left( \sum_{i=1}^{l-1} \tilde a_{i,l} \xi_i + \sum_{i=l+1}^n \tilde a_{i,l} z_i    \right)^2 
    \right)
    \\
            &+
            \frac{1}{24} \|f^{(4)} \|_{\infty} \left(\myE (h_{l}^4)+\myE (g_{l}^4)\right)
            \\
\leq &
\frac{
\left|
\myE (\xi_1^4)-1
            -
            \tau^2
\right|
}{2}
\|f^{(2)}\|_\infty
        \tilde a_{l,l}^2
            +
            \frac{
            \max\left(
    \left|\myE (\xi_1^2-1)^3\right|
            ,
12 (\myE (\xi_1^4)-1)
        \right)
            }{6} \|f^{(3)}\|_\infty
|\tilde a_{l,l}|
         \sum_{i=1}^{n} \tilde a_{i,l}^2 
         \\
            &+
            \frac{1}{24} \|f^{(4)} \|_{\infty} \left(\myE (h_{l}^4)+\myE (g_{l}^4)\right)
            .
        \end{split}
    \end{equation}
    Now we bound $\myE (h_l^4)$ and $\myE (g_l^4)$.
    By direct calculation,
\begin{equation*}
    \begin{split}
        \myE (h_l^4)
        =&
        \myE (\xi_1^2 - 1)^4 \tilde a_{l,l}^4
        + 24 \myE [ \xi_1^2(\xi_1^2 -1)^2]
        \tilde a_{l,l}^2
        \myE \left( 
        \sum_{i=1}^{l-1} \tilde a_{i,l} \xi_i 
        +\sum_{i =l +1}^n \tilde a_{i,l} z_i 
        \right)^2
        \\
        &+
        16 \myE (\xi_1^4 )
        \myE \left( 
        \sum_{i=1}^{l-1} \tilde a_{i,l} \xi_i 
        +\sum_{i =l +1}^n \tilde a_{i,l} z_i 
    \right)^4
    \\
    =&
        \myE (\xi_1^2 - 1)^4 \tilde a_{l,l}^4
        + 24 \myE [ \xi_1^2(\xi_1^2 -1)^2]
        \tilde a_{l,l}^2
    \left((\sum_{i=1}^n \tilde a_{i,l}^2) - \tilde a_{l,l}^2\right)
        \\
        &+
        16 \myE (\xi_1^4 )
    \left(
        \left(\myE (\xi_1^4) - 3\right)\sum_{i=1}^{l-1} \tilde a_{i,l}^4 
        + 3 \left( (\sum_{i=1}^n \tilde a_{i,l}^2) - \tilde a_{l,l}^2 \right)^2
    \right)
    .
    \end{split}
\end{equation*}
To upper bound the above quantity, we use the facts
$
24 \myE [ \xi_1^2(\xi_1^2 -1)^2]
\leq 
2(16\myE (\xi_1^2 -1)^4 + (9/4) \myE  (\xi_1^4) )
$,
$\myE (\xi_1^2 - 1)^4\leq \myE (\xi_1^8)$ and
\begin{equation*}
        \left(\myE (\xi_1^4) - 3\right)\sum_{i=1}^{l-1} \tilde a_{i,l}^4 
        \leq
        \left(\myE (\xi_1^4) - 1\right)\sum_{i=1}^{l-1} \tilde a_{i,l}^4 
        \leq
        \left(\myE (\xi_1^4) - 1\right)
        \left( (\sum_{i=1}^n \tilde a_{i,l}^2) - \tilde a_{l,l}^2 \right)^2
        .
\end{equation*}
Then we obtain the bound 
\begin{equation}\label{eq:Lin2}
\myE (h_l^4) \leq \left(16 \myE (\xi_1^8) + 32 \myE (\xi_1^4)\right) \left(\sum_{i=1}^n \tilde a_{i,l}^2\right)^2.
\end{equation}
Similarly, we have
\begin{equation}\label{eq:Lin3}
\myE (g_l^4) \leq \left(  48 \myE (\xi_1^4) + 3\tau^4 + 96 \right) \left(\sum_{i=1}^n \tilde a_{i,l}^2\right)^2.
\end{equation}
Combining \eqref{eq:Lin1}, \eqref{eq:Lin2} and \eqref{eq:Lin3} yields

%It can be seen that
%\begin{align*}
        %\myE \left( 
        %\sum_{i=1}^{l-1} \tilde a_{i,l} \xi_i 
        %+\sum_{i =l +1}^n \tilde a_{i,l} z_i 
        %\right)^2
        %=&\left(\sum_{i=1}^n \tilde a_{i,l}^2\right) - \tilde a_{l,l}^2
        %,
    %\end{align*}
    %and
    %\begin{align*}
        %\myE \left( 
        %\sum_{i=1}^{l-1} \tilde a_{i,l} \xi_i 
        %+\sum_{i =l +1}^n \tilde a_{i,l} z_i 
        %\right)^4
        %=&
        %\left(\myE (\xi_i^4) - 3\right)\sum_{i=1}^{l-1} \tilde a_{i,l}^4 
        %+ 3 \left( \left(\sum_{i=1}^n \tilde a_{i,l}^2\right) - \tilde a_{l,l}^2 \right)^2
        %\\
        %\leq &
        %\left(
         %\myE (\xi_i^4)
     %+ 2\right) \left( \left(\sum_{i=1}^n \tilde a_{i,l}^2\right) - \tilde a_{l,l}^2 \right)^2.
%\end{align*}
    \begin{equation*}
        \begin{split}
             &
             \sum_{l=1}^n \left| \myE f(S_{l}+h_{l})-\myE f(S_{l}+g_{l})\right|
             \\
\leq&
\frac{
\left|
\myE (\xi_1^4)-1
            -
            \tau^2
\right|
}{2}
\|f^{(2)}\|_\infty
\sum_{l=1}^n \tilde a_{l,l}^2
            +
            \frac{
            \max\left(
    \left|\myE (\xi_1^2-1)^3\right|
            ,
12 (\myE (\xi_1^4)-1)
        \right)
            }{6} \|f^{(3)}\|_\infty
            \sum_{l=1}^n 
            \left(|\tilde a_{l,l}|
         \sum_{i=1}^{n} \tilde a_{i,l}^2 
     \right)
         \\
            &+
            \frac{
             16 \myE (\xi_1^8) + 80 \myE (\xi_1^4) + 3\tau^4 + 96 
            }{24} \|f^{(4)} \|_{\infty} 
            \sum_{l=1}^n \left( \sum_{i=1}^n \tilde a_{i,l}^2 \right)^2
            .
        \end{split}
    \end{equation*}
    This completes the proof.
 
\end{proof}


\begin{proof}[\textbf{Proof of Theorem \ref{thm:criticalValue}}]
    By a standard subsequence argument, we only need to prove the theorem holds along a subsequence of $\{n\}$.
    Hence, without loss of generality, we assume $\hat \tau^2 \xrightarrow{a.s.} \phi^2 \myE (\epsilon_1^4)-1$.
    Note that almost surely, the distributions $\mathcal L (S_{\hat \tau}^*  |\hat \tau)$
    have bounded second moment and are thus tight.
    Hence, without loss of generality, we assume $\mathcal L (S_{\hat \tau}^*  |\hat \tau)$ weakly converges to a limit distribution with distribution function $F(x)$.
    Let $S^\dagger$ be a random variable with this limit distribution.
    Similar to \cite{chen2010tests}, Proposition A.1.(iii), it can be shown that $\myE [(S^*_{\hat \tau})^4|\hat \tau]$ is uniformly bounded almost surely.
    Then almost surely, $\mathcal L ((S_{\hat \tau}^*)^2  |\hat \tau)$ is  uniformly integrable.
    Hence $F(x)$ can not concentrate on a single point.
    Consequently, $F(x)$ is continous and is strict increasing for $x\in\{x:0<F(x)<1\}$; see \cite{Sevast1961A} as well as the remark made by A. N. Kolmogorov.

    For every $f\in \mathscr C^4 (\mathbb R)$,
    we have $| \myE [f(S^*_{\hat \tau}) |\hat\tau] - \myE f(S^\dagger) |\to 0$ almost surely.
    Also, Theorem \ref{approximation} implies
        $|\myE f(S)- \myE [f(S^*_{\hat \tau})|\hat\tau] |\to 0$ almost surely.
        Thus, $|\myE f(S)- \myE f(S^\dagger) |\to 0$.
        That is, $S\rightsquigarrow S^\dagger$.
        On the other hand, since
    \begin{equation*}
        \Pr\left(
            S_{\hat \tau}^*
            >
            \frac{F^{-1}_{\hat \tau} (1-\alpha)}{
            \sqrt{
    2 \mytr(\BA^2)
    +
    ( \phi^2 \myE (\epsilon_1^4)-3) \sum_{i=1}^n a_{i,i}^2
            }             
    }
\Bigg | \hat{\tau} \right)=\alpha,
    \end{equation*}
    we have almost surely,
    \begin{equation}\label{eq:slt1}
            \frac{F^{-1}_{\hat \tau} (1-\alpha)}{
            \sqrt{
    2 \mytr(\BA^2)
    +
    ( \phi^2 \myE (\epsilon_1^4)-3) \sum_{i=1}^n a_{i,i}^2
            }             
    }
    \to F^{-1}(1-\alpha).
    \end{equation}
We also need the fact that
\begin{equation}
    (\sqrt \phi \bepsilon)^\top 
    \tilde \BU_a \tilde \BU_a^\top
(\sqrt \phi \bepsilon)
=
(1+o_p(1))(n-q),
    \label{eq:slt2}
\end{equation}
which is a consequence of
\begin{equation*}
    \myE\left(  (\sqrt \phi \bepsilon)^\top 
    \tilde \BU_a \tilde \BU_a^\top
(\sqrt \phi \bepsilon)
\right)
=n-q
,\quad
    \myVar\left(  (\sqrt \phi \bepsilon)^\top 
    \tilde \BU_a \tilde \BU_a^\top
(\sqrt \phi \bepsilon)
\right)
=O(n-q)
    .
\end{equation*}
The fact $S\rightsquigarrow S^\dagger$, the equations \eqref{eq:slt1}, \eqref{eq:slt2} and Slutsky's theorem leads to
\begin{align*}
            &\Pr\left( T > \frac{F^{-1}_{\hat \tau} (1-\alpha)-\mytr\left( (\BX_b^* \BX_b^{*\top})^{-1}  \right)}{n-q} \right)
            \\
            =&\Pr
            \Bigg( 
                (\sqrt \phi \bepsilon)^\top \left( -  \tilde \BU_a (\BX_b^* \BX_b^{*\top})^{-1} \tilde \BU_a^\top 
            +\frac{
\mytr\left( (\BX_b^* \BX_b^{*\top})^{-1}  \right)
}{n-q}
\tilde \BU_a  \tilde \BU_a^\top
 \right) 
(\sqrt \phi \bepsilon)
            \\
            &
            \quad
            \quad
            >
            \frac{
            (\sqrt \phi \bepsilon)^\top \tilde \BU_a  \tilde \BU_a^\top (\sqrt \phi \bepsilon)
}{n-q} F^{-1}_{\hat \tau} (1-\alpha)
\Bigg)
            \\
            =&
            \Pr \left( \bxi^\top \BA \bxi  
                >
                \frac{
            (\sqrt \phi \bepsilon)^\top \tilde \BU_a  \tilde \BU_a^\top (\sqrt \phi \bepsilon)
}{n-q}
                F^{-1}_{\hat \tau} (1-\alpha)
            \right)
            \\
            =&
            \Pr \left( S> 
\frac{
            (\sqrt \phi \bepsilon)^\top \tilde \BU_a  \tilde \BU_a^\top (\sqrt \phi \bepsilon)
}{n-q}
            \frac{F^{-1}_{\hat \tau} (1-\alpha)}{
            \sqrt{
    2 \mytr(\BA^2)
    +
    ( \phi^2 \myE (\epsilon_1^4)-3) \sum_{i=1}^n a_{i,i}^2
            }             
    }
             \right) 
             \\
            =& 
            \Pr \Bigg( S+F^{-1}(1-\alpha)-
\frac{
            (\sqrt \phi \bepsilon)^\top \tilde \BU_a  \tilde \BU_a^\top (\sqrt \phi \bepsilon)
}{n-q}
            \frac{F^{-1}_{\hat \tau} (1-\alpha)}{
            \sqrt{
    2 \mytr(\BA^2)
    +
    ( \phi^2 \myE (\epsilon_1^4)-3) \sum_{i=1}^n a_{i,i}^2
            }             
    }
    \\
    &\quad \quad>F^{-1}(1-\alpha)
             \Bigg) 
             \\
             \to&  \alpha
             .
\end{align*}
This completes the proof.

\end{proof}



\begin{proof}[\textbf{Proof of Proposition \ref{prop:estimation}}]
    From \cite{Bai2017}, Theorem 2.1, one can obtain the explicit forms of $\myVar \left( \tilde \bepsilon^\top \left( \BI_n-\BP_a \right) \tilde \bepsilon  \right)$ and $\myVar \left( \sum_{i=1}^n \tilde \epsilon_i^4 \right)$ which involves the traces of certain matrices.
    Using \cite{book:1244195}, Theorem 5.5.1, one can see that the eigenvalues of these matrices are all bounded.
    Hence it can be deduced that $\myVar \left( \tilde \bepsilon^\top \left( \BI_n-\BP_a \right) \tilde \bepsilon  \right)=O(n)$ and $\myVar \left( \sum_{i=1}^n \tilde \epsilon_i^4 \right)=O(n)$.
    Thus,
    \begin{align*}
        & \tilde \bepsilon^\top \left( \BI_n-\BP_a \right) \tilde \bepsilon
        = (n-q) \sigma^2+O_P(\sqrt n),
        \\
        & \sum_{i=1}^n \tilde \epsilon_i^4 
        =
        3\sigma^4 \mytr (\BI_n-\BP_a)^{\circ 2} 
        +\left( \myE (\epsilon_1^4)-3 \sigma^{4}\right)
        \mytr \left( (\BI_n-\BP_a)^{\circ 2}  \right)^2 + O_P(\sqrt n).
    \end{align*}
    It follows that
    \begin{equation*}
        \hat \tau ^2 = \sigma^{-4}\myE (\epsilon_1^4)-1
        +O_P\left( \frac{\sqrt n}{\mytr \left( \left(  \BI_n - \BP_a \right)^{\circ 2} \right)^2} \right).
    \end{equation*}
    Let $\delta_{i,j}=1$ if $i=j$ and $0$ if $i\neq j$.
    We have
    \begin{equation*}
        \begin{split}
        n=&
        \sum_{i=1}^n \sum_{j=1}^n \delta_{i,j}^4 
        \\
        =&
        \sum_{i=1}^n \sum_{j=1}^n \left( \delta_{i,j}-(\BP_{a})_{i,j}+(\BP_{a})_{i,j} \right)^4 
        \\
        \leq&
    8\sum_{i=1}^n \sum_{j=1}^n \left( \delta_{i,j}-(\BP_{a})_{i,j}\right)^4+
    8\sum_{i=1}^n \sum_{j=1}^n(\BP_{a})_{i,j}^4
        \\
        \leq&
    8\sum_{i=1}^n \sum_{j=1}^n \left( \delta_{i,j}-(\BP_{a})_{i,j}\right)^4+
    8\sum_{i=1}^n \sum_{j=1}^n(\BP_{a})_{i,j}^2
    \\
    =&
        8\mytr \left( \left(  \BI_n - \BP_a \right)^{\circ 2} \right)^2
        +8q.
        \end{split}
    \end{equation*}
    Then
    \begin{equation*}
        \frac{\sqrt n}{\mytr \left( \left(  \BI_n - \BP_a \right)^{\circ 2} \right)^2} 
        =
        O\left( \frac{1}{\sqrt n} \right).
    \end{equation*}
    This completes the proof.
\end{proof}


\section{hahayy}

\begin{proof}[\textbf{Proof of Proposition \ref{Lemma:normal}}]

    Without loss of generality, we assume $\BA$ is a diagonal matrix and $|b_1|\geq \cdots\geq |b_n|$.
    By a standard subsequence argument, we only need to prove the result along a subsequence.
    Hence we can assume $\lim_{n\to \infty}\|b\|^2/\mytr(\BA^2) =c \in [0,+\infty]$.
    If $ c=0$, Lyapunov central limit theorem implies that
    \begin{equation*}
        \frac{Z^\top \BA Z + b^\top Z - \mytr(\BA)}{\sqrt{2\mytr(\BA^2)+\|\Bb\|^2}}
        =(1+o_P(1))
        \frac{Z^\top \BA Z - \mytr(\BA)}{\sqrt{2\mytr(\BA^2)}}
        +o_P(1)
        \rightsquigarrow \mathcal N (0,1).
    \end{equation*}
    If $c=+\infty$,
    \begin{equation*}
        \frac{Z^\top \BA Z + b^\top Z - \mytr(\BA)}{\sqrt{2\mytr(\BA^2)+\|\Bb\|^2}}
        =(1+o_P(1))
        \frac{b^\top Z}{\|\Bb\|}
        +o_P(1)
        \rightsquigarrow \mathcal N (0,1).
    \end{equation*}
    In what follows, we assume $c\in (0,+\infty)$.
    By Helly selection theorem, we can assume $\lim_{n\to \infty} |b_i|/\|\Bb\|= b_i^*\in [0,1]$, $i=1,2,\ldots$.
From Fatou's lemma, we have $\sum_{i=1}^{\infty} (b_i^{*})^2\leq 1$.
Consequently, $\lim_{i\to \infty} b_i^* =0$.

Note that the condition $\mytr(\BA^4)/\mytr^2 (\BA^2)\to 0$ is equivalent to 
$\lambda_1(\BA^2)/\mytr(\BA^2)\to 0$.
Then for every fixed integer $r>0$,
\begin{equation*}
    \frac{
        \sum_{i=1}^r a_{i,i}^2 
    }{
        \sum_{i=1}^n a_{i,i}^2 
    }
    \leq
    \frac{
        r \max_{1\leq i\leq n} a_{i,i}^2 
    }{
        \sum_{i=1}^n a_{i,i}^2 
    }
    \to 0.
\end{equation*}
Then there exists a sequence of positive integers $r(n)\to \infty$ such that   
    ${
        \left( 
        \sum_{i=1}^{r(n)} a_{i,i}^2 
        \right)
    }/{
        \left( 
        \sum_{i=1}^n a_{i,i}^2 
        \right)
    }
    \to 0$ and $r(n)/n\to 0$.
    Write
    \begin{equation*}
        Z^\top \BA Z + b^\top Z - \mytr(\BA)
        =
        \sum_{i=1}^{r(n)} a_{i,i}(z_i^2-1)
        +
        \sum_{i=1}^{r(n)} b_i z_i
        +
        \sum_{i=r(n)+1}^n
        \left( 
        a_{i,i}(z_i^2-1) + b_i z_i
    \right),
    \end{equation*}
    which is a sum of independent random variables.
    The first term is negligible since $\myVar ( 
        \sum_{i=1}^{r(n)} a_{i,i}(z_i^2-1)
    )=o(\sum_{i=1}^n a_{i,i}^2)$.
    Now we deal with the third term.
    From Berry-Esseen inequality (See, e.g., \cite{book:336898}, Theorem 11.2), there exists an absolute constant $C^*>0$, such that
    \begin{equation*}
        \begin{split}
        &\sup_{x\in \mathbb R}\left|
        \Pr\left( 
        \frac{
            \sum_{i=r(n)+1}^n
        \left( 
        a_{i,i}(z_i^2-1) + b_i z_i
    \right)
}{
    \sqrt{2\sum_{i=r(n)+1}^n a_{i,i}^2 + \sum_{i=r(n)+1}^n b_{i}^2}
}
\leq x
    \right)
    -\Phi(x)
    \right|
    \leq
    C^*
    \frac{
        \sum_{i=r(n)+1}^n
        \myE
        \left| 
        a_{i,i}(z_i^2-1) + b_i z_i
    \right|^3
    }{
        \left( 2\sum_{i=r(n)+1}^n a_{i,i}^2 + \sum_{i=r(n)+1}^n b_{i}^2 \right)^{3/2}
    }
    .
        \end{split}
    \end{equation*}
    By some simple algebra,
    there exist absolute constants $C_1^*,C_2^*>0$ such that for sufficiently large $n$,
    \begin{equation*}
        \begin{split}
        &\sup_{x\in \mathbb R}\left|
        \Pr\left( 
        \frac{
            \sum_{i=r(n)+1}^n
        \left( 
        a_{i,i}(z_i^2-1) + b_i z_i
    \right)
}{
    \sqrt{2\sum_{i=r(n)+1}^n a_{i,i}^2 + \sum_{i=r(n)+1}^n b_{i}^2}
}
\leq x
    \right)
    -\Phi(x)
    \right|
    \leq
    C_1^*
    \frac{
        \max_{1\leq i \leq n}
        |a_{i,i}|
    }{
        \sqrt{ \sum_{i=1}^n a_{i,i}^2 }
    }
    +
    C_2^*
    \frac{
        |b_{r(n)+1}|
    }
    {
        \|\Bb\|
    }
    .
        \end{split}
    \end{equation*}
    Since the right hand side tends to $0$, we have
    \begin{equation*}
        \frac{
            \sum_{i=r(n)+1}^n
        \left( 
        a_{i,i}(z_i^2-1) + b_i z_i
    \right)
}{
    \sqrt{2\sum_{i=r(n)+1}^n a_{i,i}^2 + \sum_{i=r(n)+1}^n b_{i}^2}
}
\rightsquigarrow \mathcal N(0,1).
    \end{equation*}
    Note that
    $
            \sum_{i=r(n)+1}^n
        \left( 
        a_{i,i}(z_i^2-1) + b_i z_i
    \right)
    $ is independent of $\sum_{i=1}^{r(n)} b_{i} z_i$ and $\sum_{i=1}^{r(n)} b_i z_i\sim \mathcal N(0,\sum_{i=1}^{r(n)}b_i^2)$.
    Thus,
    \begin{equation*}
        \frac{
\sum_{i=1}^{r(n)} b_i z_i+
            \sum_{i=r(n)+1}^n
        \left( 
        a_{i,i}(z_i^2-1) + b_i z_i
    \right)
}{
    \sqrt{2\sum_{i=1}^n a_{i,i}^2 + \sum_{i=1}^n b_{i}^2}
}
\rightsquigarrow \mathcal N(0,1).
    \end{equation*}
    This completes the proof.

\end{proof}





Note that under the normality, $T_n- \mytr ((\BX_b^* \BX_b^{*\top})^{-1})/(n-q)$ has zero mean.




\begin{proof}[\textbf{Proof of Theorem \ref{generalTheorem}}]

We note that
\begin{equation}
    \begin{split}
    &\Pr\left( 
        \frac{
            \By^{*\top} \left( \BX_b^* \BX_b^{*\top} \right)^k \By^*
        }{
            \By^{*\top} \By^*
        } 
        \leq 
        \myE (\gamma_I^k)
        +\sqrt{
            \frac{2\myVar\left( \gamma_I^k \right)}{n-q} 
        }
        x
    \right) 
    \\
    =&
    \Pr\left( 
            \By^{*\top} \left( \BX_b^* \BX_b^{*\top} \right)^k \By^*
        \leq 
        \left( 
            \myE (\gamma_I^k)
        +\sqrt{
            \frac{2\myVar\left( \gamma_I^k \right)}{n-q} 
        }
        x
        \right)
            \By^{*\top} \By^*
    \right) 
    \\
    =&
    \Pr\left( 
            \By^{*\top}
            \BB
            \By^*
            \leq 0
    \right) 
    ,
    \end{split}
    \label{eq:to2}
\end{equation}
where
\begin{equation*}
   \BB= 
            \left( \BX_b^* \BX_b^{*\top} \right)^k 
        -
        \left( 
            \myE (\gamma_I^k)
        +\sqrt{
            \frac{2\myVar\left( \gamma_I^k \right)}{n-q} 
        }
        x
        \right)
        \BI_{n-q}.
\end{equation*}
Since $\By^{*\top} \BB \By^*= \bepsilon^\top \tilde{\BU}_a \BB \tilde{\BU}_a^\top\bepsilon + 2\bepsilon^\top \tilde{\BU}_a \BB \BX_b^* \bbeta_b + \bbeta_b^{\top} \BX_b^{*\top} \BB \BX_b^* \bbeta_b$, we have
\begin{equation*}
    \begin{split}
     &\Pr\left( 
            \By^{*\top}
            \BB
            \By^*
            \leq 0
    \right) 
    \\
    =&
    \Pr\left( 
        \frac{
    \bepsilon^\top \tilde{\BU}_a \BB \tilde{\BU}_a^\top\bepsilon + 2\bepsilon^\top \tilde{\BU}_a \BB \BX_b^* \bbeta_b 
    -\phi^{-1}\mytr(\BB)
}{
    \sqrt{
        2\phi^{-2}\mytr(\BB^2)
        +4 \phi^{-1}
        \bbeta_b^\top
        \BX_b^{*\top}
        \BB^2
        \BX_b^*
        \bbeta_b
    }
}
    \leq
    \frac{
        -\bbeta_b^{\top} \BX_b^{*\top} \BB \BX_b^* \bbeta_b
        -\phi^{-1}\mytr(\BB)
    }{
    \sqrt{
        2\phi^{-2}
        \mytr(\BB^2)
        +4\phi^{-1}
        \bbeta_b^\top
        \BX_b^{*\top}
        \BB^2
        \BX_b^*
        \bbeta_b
    }
    }
\right).
    \end{split}
\end{equation*}
To apply proposition \ref{Lemma:normal}, we need to verify the condition $\lambda_1\left( \BB^2 \right)/\mytr\left(  \BB^2 \right)\to 0$.
It is straightforward to show that
    $\mytr(\BB^2) =  ( n-q +2x^2 ) \myVar ( \gamma_I^k )$.
    On the other hand,
\begin{equation*}
    \lambda_1\left( \BB^2 \right) 
    =\max_{1\leq i \leq n-q}
    \left( 
    \gamma_i^k
        -
            \myE (\gamma_I^k)
        -
        \sqrt{
            \frac{2\myVar\left( \gamma_I^k \right)}{n-q} 
        }
        x
    \right)^2
    \leq
    2
    \max_{1\leq i \leq n-q}
    \left( 
    \gamma_i^k
        -
            \myE (\gamma_I^k)
    \right)^2
        +
        4
            \frac{\myVar\left( \gamma_I^k \right)}{n-q} 
        x^2
        .
\end{equation*}
Thus,
\begin{equation*}
    \begin{split}
    \frac{
        \lambda_1(\BB^2)
    }{
        \mytr(\BB^2)
    } 
    \leq
    &
    2
    \frac{
        \max_{1\leq i \leq n-q}
        \left( 
        \gamma_i^k
            -
                \myE (\gamma_I^k)
        \right)^2
    }{
        (n-q+2x^2) \myVar (\gamma_I^k)
    }
    +4\frac{
        x^2
    }{
        (n-q)(n-q+x^2)
    }
    \\
    \leq&
    2
    \frac{
        \max_{1\leq i \leq n-q}
        \left( 
        \gamma_i^k
            -
                \myE (\gamma_I^k)
        \right)^2
    }{
        (n-q) \myVar (\gamma_I^k)
    }
    +\frac{
        4
    }{
        (n-q)
    },
    \end{split}
\end{equation*}
which tends to $0$ by the condition \eqref{eq:toBeCondition}.
Hence Proposition \ref{Lemma:normal} implies that
\begin{equation}\label{eq:to3}
    \Pr\left( 
            \By^{*\top}
            \BB
            \By^*
            \leq 0
    \right) 
=
    \Phi\left( 
    \frac{
        -\bbeta_b^{\top} \BX_b^{*\top} \BB \BX_b^* \bbeta_b
        -\phi^{-1} \mytr(\BB)
    }{
    \sqrt{
        2\phi^{-2}
        \mytr(\BB^2)
        +4\phi^{-1}
        \bbeta_b^\top
        \BX_b^{*\top}
        \BB^2
        \BX_b^*
        \bbeta_b
    }
    }
\right)
+o(1)
.
\end{equation}
Then the conclusion follows from \eqref{eq:to2}, \eqref{eq:to3} and the following facts
\begin{align*}
    \mytr(\BB)
    &=
    -(n-q)
     \sqrt{
         \frac{
             2\myVar\left( \gamma_I^k \right)
         }{
             n-q
         }
    } 
    x,
    \\
    \mytr(\BB^2) &= (1+o(1)) ( n-q ) \myVar ( \gamma_I^k ),
    \\
        \bbeta_b^\top
        \BX_b^{*\top}
        \BB
        \BX_b^*
        \bbeta_b
        &= 
        (n-q)\left( 
            \myCov\left( \gamma_I^k, \gamma_I w_I^2 \right)
            -
            \myE (\gamma_I w_I^2)
            \sqrt{\frac{2\myVar\left( \gamma_I^k \right)}{n-q}} 
            x
        \right),
    \\
        \bbeta_b^\top
        \BX_b^{*\top}
        \BB^2
        \BX_b^*
        \bbeta_b
    &=
    (n-q) \myE\left[ 
        \left( \gamma_I^k -\myE(\gamma_I^k) -\sqrt{\frac{2\myVar (\gamma_I^k)}{n-q}}x \right)^2
        \gamma_I w_I^2
    \right]
    .
\end{align*}



\end{proof}





 



\end{appendices}



\bibliographystyle{apalike}
\bibliography{mybibfile}



\end{document}
