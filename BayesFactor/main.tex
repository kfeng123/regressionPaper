\documentclass[11pt]{article}
 
\newcommand\CG[1]{\textcolor{red}{#1}}

\usepackage{lineno,hyperref}

\usepackage[margin=1 in]{geometry}
\renewcommand{\baselinestretch}{1.25}


%\usepackage{refcheck}
\usepackage{authblk}
\usepackage{galois} % composition function \comp
\usepackage{bm}
\usepackage{amsmath}
\usepackage{amssymb}
\usepackage{mathrsfs}
\usepackage{amsthm}
\usepackage{natbib}
\usepackage{graphicx}
\usepackage{color}
\usepackage{booktabs}
\usepackage[page,title]{appendix}
%\renewcommand\appendixname{haha}
\usepackage{enumerate}
\usepackage{changepage}
\usepackage{datetime}
\newdate{date}{9}{1}{2017}

%%%%%%%%%%%%%%  Notations %%%%%%%%%%
\DeclareMathOperator{\mytr}{tr}
\DeclareMathOperator{\mydiag}{diag}
\DeclareMathOperator{\myrank}{Rank}
\DeclareMathOperator{\myP}{P}
\DeclareMathOperator{\myE}{E}
\DeclareMathOperator{\myVar}{Var}
\DeclareMathOperator*{\argmax}{arg\,max}
\DeclareMathOperator*{\argmin}{arg\,min}


\newcommand{\Ba}{\mathbf{a}}    \newcommand{\Bb}{\mathbf{b}}    \newcommand{\Bc}{\mathbf{c}}    \newcommand{\Bd}{\mathbf{d}}    \newcommand{\Be}{\mathbf{e}}    \newcommand{\Bf}{\mathbf{f}}    \newcommand{\Bg}{\mathbf{g}}    \newcommand{\Bh}{\mathbf{h}}    \newcommand{\Bi}{\mathbf{i}}    \newcommand{\Bj}{\mathbf{j}}    \newcommand{\Bk}{\mathbf{k}}    \newcommand{\Bl}{\mathbf{l}}
\newcommand{\Bm}{\mathbf{m}}    \newcommand{\Bn}{\mathbf{n}}    \newcommand{\Bo}{\mathbf{o}}    \newcommand{\Bp}{\mathbf{p}}    \newcommand{\Bq}{\mathbf{q}}    \newcommand{\Br}{\mathbf{r}}    \newcommand{\Bs}{\mathbf{s}}    \newcommand{\Bt}{\mathbf{t}}    \newcommand{\Bu}{\mathbf{u}}    \newcommand{\Bv}{\mathbf{v}}    \newcommand{\Bw}{\mathbf{w}}    \newcommand{\Bx}{\mathbf{x}}
\newcommand{\By}{\mathbf{y}}    \newcommand{\Bz}{\mathbf{z}}    
\newcommand{\BA}{\mathbf{A}}    \newcommand{\BB}{\mathbf{B}}    \newcommand{\BC}{\mathbf{C}}    \newcommand{\BD}{\mathbf{D}}    \newcommand{\BE}{\mathbf{E}}    \newcommand{\BF}{\mathbf{F}}    \newcommand{\BG}{\mathbf{G}}    \newcommand{\BH}{\mathbf{H}}    \newcommand{\BI}{\mathbf{I}}    \newcommand{\BJ}{\mathbf{J}}    \newcommand{\BK}{\mathbf{K}}    \newcommand{\BL}{\mathbf{L}}
\newcommand{\BM}{\mathbf{M}}    \newcommand{\BN}{\mathbf{N}}    \newcommand{\BO}{\mathbf{O}}    \newcommand{\BP}{\mathbf{P}}    \newcommand{\BQ}{\mathbf{Q}}    \newcommand{\BR}{\mathbf{R}}    \newcommand{\BS}{\mathbf{S}}    \newcommand{\BT}{\mathbf{T}}    \newcommand{\BU}{\mathbf{U}}    \newcommand{\BV}{\mathbf{V}}    \newcommand{\BW}{\mathbf{W}}    \newcommand{\BX}{\mathbf{X}}
\newcommand{\BY}{\mathbf{Y}}    \newcommand{\BZ}{\mathbf{Z}}    

\newcommand{\bfsym}[1]{\ensuremath{\boldsymbol{#1}}}

 \def\balpha{\bfsym \alpha}
 \def\bbeta{\bfsym \beta}
 \def\bgamma{\bfsym \gamma}             \def\bGamma{\bfsym \Gamma}
 \def\bdelta{\bfsym {\delta}}           \def\bDelta {\bfsym {\Delta}}
 \def\bfeta{\bfsym {\eta}}              \def\bfEta {\bfsym {\Eta}}
 \def\bmu{\bfsym {\mu}}                 \def\bMu {\bfsym {\Mu}}
 \def\bnu{\bfsym {\nu}}
 \def\btheta{\bfsym {\theta}}           \def\bTheta {\bfsym {\Theta}}
 \def\beps{\bfsym \varepsilon}          \def\bepsilon{\bfsym \varepsilon}
 \def\bsigma{\bfsym \sigma}             \def\bSigma{\bfsym \Sigma}
 \def\blambda {\bfsym {\lambda}}        \def\bLambda {\bfsym {\Lambda}}
 \def\bomega {\bfsym {\omega}}          \def\bOmega {\bfsym {\Omega}}
 \def\brho   {\bfsym {\rho}}
 \def\btau{\bfsym {\tau}}
 \def\bxi{\bfsym {\xi}}
 \def\bzeta{\bfsym {\zeta}}
% May add more in future.
%%%%%%%%%%%%%%%%%%%%%%%%%%%%%%%%%%%%



\theoremstyle{plain}
\newtheorem{theorem}{\quad\quad Theorem}
\newtheorem{proposition}{\quad\quad Proposition}
\newtheorem{corollary}{\quad\quad Corollary}
\newtheorem{lemma}{\quad\quad Lemma}
\newtheorem{example}{Example}
\newtheorem{assumption}{\quad\quad Assumption}
\newtheorem{condition}{\quad\quad Condition}

\theoremstyle{definition}
\newtheorem{definition}{\quad\quad Definition}
\newtheorem{remark}{\quad\quad Remark}
\theoremstyle{remark}



\title{Bayes factors for linear regression}



\author[1]{Rui Wang}
%\author[2]{xx}
%\author[1,3]{Xingzhong Xu\thanks{Corresponding author\\Email address: xuxz@bit.edu.cn}}
%\affil[1]{
%School of Mathematics and Statistics, Beijing Institute of Technology, Beijing 
    %100081,China
%}
%\affil[2]{
    %xx
%}
%\affil[3]{
%Beijing Key Laboratory on MCAACI, Beijing Institute of Technology, Beijing 100081,China
%}



\begin{document}
\maketitle
\section{Introduction}
This note gives a review for Bayes factors for linear regression.
\section{Mixture of $g$ prior}
This section is adapted from \cite{Liang2008Mixtures}.
Suppose $\BY\in \mathbb R^n$ is generated from the model
\begin{equation*}
    \mathcal M_{\gamma}: \BY=\mathbf 1_n \alpha + \BX \bbeta+\bepsilon,
\end{equation*}
where $\BX \in \mathbb R^{n\times p}$ and $\bepsilon\sim \mathcal N (0, \phi^{-1} \BI_n)$.

Let $\BX_{\gamma}\in \mathbb R^{n\times p_{\gamma}}$ be a submatrix of $\BX$.
Then the submodel $\mathcal M_\gamma$ is defined as 
\begin{equation*}
    \mathcal M_{\gamma}: \BY=\mathbf 1_n \alpha + \BX_\gamma \bbeta_{\gamma}+\bepsilon.
\end{equation*}
The null model $\mathcal M_N $ is 
\begin{equation*}
    \mathcal M_{\gamma}: \BY=\mathbf 1_n \alpha +\bepsilon.
\end{equation*}
We would like to compare $\mathcal M_\gamma$ with $\mathcal M_N$.
Without loss of generality, we assume $\mathbf 1_n^\top \BX_\gamma=0$.
Under $\mathcal M_N$, the $g$ prior is
\begin{equation*}
    p(\alpha,\phi|\mathcal M_N) = \frac{1}{\phi}.
\end{equation*}
Under $\mathcal M_\gamma$, the $g$ prior is
\begin{equation*}
    \bbeta_{\gamma}|\phi,\mathcal M_\gamma \sim \mathcal N (0,\frac{g}{\phi} (\BX_{\gamma}^\top \BX_{\gamma})^{-1}),
    \quad p(\alpha|\phi,\mathcal M_\gamma)\propto 1,
    \quad p(\phi|\mathcal M_\gamma) = \frac{1}{\phi}.
\end{equation*}
The joint pdf is
\begin{equation*}
    \begin{split}
    &p(\BY,\alpha,\bbeta_{\gamma},\phi|\mathcal M_\gamma)
    =
    p(\BY|\alpha,\bbeta_\gamma,\phi,\mathcal M_\gamma)
    p(\bbeta_\gamma |\phi,\mathcal M_\gamma)
    p(\alpha |\phi,\mathcal M_\gamma)
    p(\phi |\mathcal M_\gamma)
    \\
    =&
    (2\pi)^{-(n+p_\gamma)/{2}} g^{-{p_\gamma}/{2}} \phi^{{(n+p_\gamma)}/{2}-1} |\BX_\gamma^\top \BX_\gamma|^{1/2}
    \exp\left\{-\frac{n\phi}{2}(\bar{\BY}-\alpha)^2\right\}
    \\
    &\exp\left\{
        -\frac{\phi (g+1)}{2g}\left\| \BX_\gamma \left(\bbeta_\gamma-\frac{g}{g+1}\hat{\bbeta}_\gamma\right)\right\|^2
        -\frac{\phi}{2(g+1)} \left\|\BX_\gamma \hat{\bbeta}_\gamma\right\|^2
        -\frac{\phi}{2}\left\|\BY-\mathbf 1_n \bar \BY- \BX_\gamma \hat{\bbeta}_\gamma\right\|^2
    \right\},
    \end{split}
\end{equation*}
where $\bar \BY= n^{-1}\mathbf 1_n^\top \BY$,
$\hat{\bbeta}_\gamma =(\BX_\gamma^\top \BX_\gamma)^{-1} \BX_\gamma^\top \BY $.

Direct calculation yields
\begin{equation*}
    p(\BY| \mathcal M_\gamma,g)
    =
    \frac{\Gamma((n-1)/2)}{\pi^{(n-1)/2}\sqrt{n}}
    \left\|\BY-\mathbf 1_n \bar \BY\right\|^{-(n-1)}
    \frac{(1+g)^{(n-p_{\gamma}-1)/2}}{[1+g(1-R_\gamma^2)]^{(n-1)/2}},
\end{equation*}
where $R_\gamma^2 = 1-\|\BY -\mathbf 1_n \bar \BY- \BX_\gamma \hat{\bbeta}_\gamma\|^2/\|\BY-\mathbf 1_n \bar \BY\|^2$.
Also, we have
\begin{equation*}
    p(\BY| \mathcal M_N)
    =
    \frac{\Gamma((n-1)/2)}{\pi^{(n-1)/2}\sqrt{n}}
    \left\|\BY-\mathbf 1_n \bar \BY\right\|^{-(n-1)}.
\end{equation*}
Thus,
\begin{equation*}
    \text{BF}[\mathcal M_\gamma :\mathcal M_N]=
    (1+g)^{(n-p_\gamma-1)/2}
    [1+g(1-R^2_\gamma)]^{-(n-1)/2}.
\end{equation*}

\subsection{Choices of $g$}
\paragraph{Local empirical Bayes.} The local EB estimates a separate $g$ for each model $\mathcal M_\gamma$.
\begin{equation*}
    \hat g_\gamma^{\text{EBL}} =
    \argmax_{g\geq 0} p(\BY | \mathcal M_\gamma, g) 
    =
    \argmax_{g\geq 0} \frac{(1+g)^{(n-p_{\gamma}-1)/2}}{[1+g(1-R_\gamma^2)]^{(n-1)/2}}
    =\max\{F_\gamma-1, 0\},
\end{equation*}
where 
\begin{equation*}
F_\gamma=
\frac{R^2_\gamma/p_\gamma}{
    (1-R^2_\gamma) / (n-1-p_\gamma)
}
\end{equation*}
is the usual $F$ statistic for testing $\bbeta_\gamma=0$.

\paragraph{Global empirical Bayes.} The global EB procedure assumes one common $g$ for all models.
\begin{equation*}
    \hat g_\gamma^{\text{EBG}} =
    \argmax_{g\geq 0} \sum_{\gamma} p(\mathcal M_\gamma) p(\BY | \mathcal M_\gamma, g) 
    =
    \argmax_{g\geq 0} \sum_{\gamma} p(\mathcal M_\gamma) \frac{(1+g)^{(n-p_{\gamma}-1)/2}}{[1+g(1-R_\gamma^2)]^{(n-1)/2}}.
\end{equation*}







%\begin{appendices}
    %\section{haha1}
    %\section{haha2}
%\end{appendices}
%\section*{Acknowledgements}
%This work was supported by the National Natural Science Foundation of China under Grant Nos.\ xxxxx, xxxx.



\bibliographystyle{apalike}
\bibliography{mybibfile}

\end{document}
