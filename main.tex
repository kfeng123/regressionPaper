\documentclass[review]{elsarticle}

\usepackage{lineno,hyperref}
\usepackage{amsmath,amssymb,amsthm}
\modulolinenumbers[5]

\journal{Journal of \LaTeX\ Templates}

%%%%%%%%%%%%%%%%%%%%%%%
%% Elsevier bibliography styles
%%%%%%%%%%%%%%%%%%%%%%%
%% To change the style, put a % in front of the second line of the current style and
%% remove the % from the second line of the style you would like to use.
%%%%%%%%%%%%%%%%%%%%%%%

%% Numbered
%\bibliographystyle{model1-num-names}

%% Numbered without titles
%\bibliographystyle{model1a-num-names}

%% Harvard
%\bibliographystyle{model2-names.bst}\biboptions{authoryear}

%% Vancouver numbered
%\usepackage{numcompress}\bibliographystyle{model3-num-names}

%% Vancouver name/year
%\usepackage{numcompress}\bibliographystyle{model4-names}\biboptions{authoryear}

%% APA style
%\bibliographystyle{model5-names}\biboptions{authoryear}

%% AMA style
%\usepackage{numcompress}\bibliographystyle{model6-num-names}

%% `Elsevier LaTeX' style
\bibliographystyle{elsarticle-num}
%%%%%%%%%%%%%%%%%%%%%%%
\theoremstyle{plain}
\newtheorem{theorem}{\quad\quad Theorem}
\newtheorem{proposition}{\quad\quad Proposition}
\newtheorem{corollary}[theorem]{Corollary}
\newtheorem{lemma}{Lemma}
\newtheorem{example}{Example}
\newtheorem{assumption}{\quad\quad Assumption}
\newtheorem{condition}{Condition}
\theoremstyle{definition}
\newtheorem{remark}{\quad\quad Remark}
\theoremstyle{remark}
\begin{document}

\begin{frontmatter}

\title{Elsevier \LaTeX\ template\tnoteref{mytitlenote}}
\tnotetext[mytitlenote]{Fully documented templates are available in the elsarticle package on \href{http://www.ctan.org/tex-archive/macros/latex/contrib/elsarticle}{CTAN}.}

%% Group authors per affiliation:
\author{Elsevier\fnref{myfootnote}}
\address{Radarweg 29, Amsterdam}
\fntext[myfootnote]{Since 1880.}

%% or include affiliations in footnotes:
\author[mymainaddress,mysecondaryaddress]{Elsevier Inc}
\ead[url]{www.elsevier.com}

\author[mysecondaryaddress]{Global Customer Service\corref{mycorrespondingauthor}}
\cortext[mycorrespondingauthor]{Corresponding author}
\ead{support@elsevier.com}

\address[mymainaddress]{1600 John F Kennedy Boulevard, Philadelphia}
\address[mysecondaryaddress]{360 Park Avenue South, New York}

\begin{abstract}
This template helps you to create a properly formatted \LaTeX\ manuscript.
\end{abstract}

\begin{keyword}
\texttt{elsarticle.cls}\sep \LaTeX\sep Elsevier \sep template
\MSC[2010] 00-01\sep  99-00
\end{keyword}

\end{frontmatter}

\linenumbers

\section{Main}

Suppose $(X_1^T,Y_1)$,\ldots,$(X_n^T, Y_n)$ are i.i.d.\ from $N_{p+1}(\mu,\Sigma)$, where $X_i\in \mathbb{R}^p$ and $Y_i\in \mathbb{R}$. Denote $X=(X_1,\ldots,X_n)$, $Y=(Y_1,\ldots,Y_n)^T$.

Write $Y=\beta_0 \textbf{1}_n+X^T \beta+\epsilon$, where $\textbf{1}_n$ is $n$ dimensional vector with all elements equal to $1$. $\epsilon$ has distribution $N(0,\sigma^2 I_n)$.

The problem is to test hypotheses $H: \beta=0$.

The test statistic is

\[
    T=\frac{
        (\textbf{1}_n^T(X^T X)^{-1}Q_n Y)^2
    }{
        \hat{\sigma}^2
        \textbf{1}_n^T(X^T X)^{-1}Q_n (X^T X)^{-1}\textbf{1}_n
    }.
\]
where $Q_n=I_n-\frac{1}{n}\textbf{1}_n\textbf{1}_n^T$ and
\[
    \hat{\sigma}^2=\frac{1}{n-2} Y^T Q_n\Big[
        I_n-\frac{(X^T X)^{-1}\textbf{1}_n\textbf{1}_n^T(X^T X)^{-1}}{
        \textbf{1}_n^T(X^T X)^{-1}Q_n(X^T X)^{-1}\textbf{1}_n
        }
        \Big]Q_n Y
\] 


Let $Q_n=WW^T$ be the rank decomposition of $Q_n$, where $W_n$ is a $n\times n-1$ matrix with $W^T W=I_{n-1}$. The new test statistic is

\[
    T=\frac{y^T Q_n y}{
        y^T W(W^T X^T X W)^{-1}W^T y
    } 
    \]
or equivalently
\[
    \frac{y^T Q_n y}{
        y^T Q_n(X^T X)^{-1}Q_n y-(y^T Q_n(X^T X)^{-1}\textbf{1}_n)^2/(\textbf{1}_n(X^T X)^{-1}\textbf{1}_n)
    } 
    \]


Let $\tilde{y}=W^T y$, $\tilde{X}=XW$, $\tilde{\epsilon}=W^T \epsilon$. Then
\[
    \tilde{y}=\tilde{X}^T \beta + \tilde{\epsilon}
    \]
and
\[
    T=\frac{\tilde{y}^T \tilde{y}}{
        \tilde{y}^T{(\tilde{X}^T \tilde{X})}^{-1}\tilde{y}
    }
    \]
Next we derive another form of of $T$. We follow the similar technique of Hotelling's $T^2$.


Let $R$ be an $(n-1)\times (n-1)$ orthogonal matrix satisfies 
\[
    R\tilde{y}=
    \begin{pmatrix}
        \|\tilde{y}\|\\
        0\\
        \ldots\\
        0
    \end{pmatrix}.
    \]
%Note that $R$ is independent of $\tilde{X}$ under null hypotheses since it only depends on $\tilde{y}$.
We can write
    \begin{equation}\label{Tformula}
        \begin{aligned}
            T&=\frac{\|\tilde{y}\|^2
        }{\tilde{y}R^T{(R\tilde{X}^T\tilde{X}R^T)}^{-1}R\tilde{y}}\\
            %&={(R\tilde{X}^T \tilde{X} R^T)}_{11\cdot 2}.
        \end{aligned}
    \end{equation}
Denote by $B=R\tilde{X}^T \tilde{X} R^T$, then
\[
    T=\frac{1}{(B^{-1})_{11}}.
    \]

Let
\[
    B=\begin{pmatrix} 
        b_{11} & b_{(1)}^T\\
        b_{(1)} & B_{22}
    \end{pmatrix},
    \]

and apply the matrix inverse formula, we have $(B^{-1})_{11}=1/(b_{11}-b_{(1)}^T B_{22}^{-1}b_{(1)})$. Hence

\[
   T= b_{11}-b_{(1)}^T B_{22}^{-1}b_{(1)}.
    \]
\section{Asymptotic distribution}
Note that conditioning on $\tilde{y}$, $R$ is a constant orthogonal matrix.
And $\tilde{y}$ is independent of $\tilde{X}$ under null hypotheses.
So  $B|\tilde{y}$ has the same distribution with $\tilde{X}^T \tilde{X}$ under null hypotheses.
Hence $B$ is independent of $\tilde{y}$ and can be written as
    \begin{equation}\label{Xdis}
    B=\sum_{i=1}^p \lambda_i z_i z_i^T
    \end{equation}
where $z_i$'s are i.i.d.\ $n-1$ dimensional random vectors  distributed as $N(0,I_{n-1})$, $\lambda_1\geq \lambda_2\ldots \geq \lambda_p>0$ are eigenvalues of $\Sigma_X$. 
Denote by $\Lambda=\textrm{diag} (\lambda_1,\ldots,\lambda_p)$, $Z=(Z_1,\ldots,Z_p)$. Let $Z_{(1)}$ and $Z_{(2)}$ be the first $1$ row and last $n-2$ rows of $Z$, that is
\[
    Z=\begin{pmatrix} 
        Z_{(1)}\\
        Z_{(2)}
    \end{pmatrix}.
    \]
Then
\begin{equation}
    \begin{aligned}
        B&=Z\Lambda Z^T\\
        &=\begin{pmatrix}
            Z_{(1)}\Lambda Z_{(1)}^T & Z_{(1)}\Lambda Z_{(2)}^T\\
            Z_{(2)}\Lambda Z_{(1)}^T & Z_{(2)}\Lambda Z_{(2)}^T\\
        \end{pmatrix}.
    \end{aligned}
\end{equation}
Hence

\begin{equation}
    \begin{aligned}
        T&=Z_{(1)}\Lambda Z_{(1)}^T-Z_{(1)}\Lambda Z_{(2)}^T{(Z_{(2)}\Lambda Z_{(2)}^T)}^{-1}Z_{(2)}\Lambda Z_{(1)}^T\\
        &=Z_{(1)}\big(\Lambda -\Lambda Z_{(2)}^T{(Z_{(2)}\Lambda Z_{(2)}^T)}^{-1}Z_{(2)}\Lambda \big)Z_{(1)}^T.\\
    \end{aligned}
\end{equation}
But
    \begin{equation}
        \begin{aligned}
    \textrm{rank}(\Lambda Z_{(2)}^T{(Z_{(2)}\Lambda Z_{(2)}^T)}^{-1}Z_{(2)}\Lambda)
            &=\textrm{rank}(\Lambda^{\frac{1}{2}} Z_{(2)}^T{(Z_{(2)}\Lambda Z_{(2)}^T)}^{-1}Z_{(2)}\Lambda^{\frac{1}{2}})\\
            &=\textrm{rank}(I_{n-2})=n-2,\\
        \end{aligned}
    \end{equation}
and
    \begin{equation}
        \begin{aligned}
    \textrm{rank}(\Lambda-\Lambda Z_{(2)}^T{(Z_{(2)}\Lambda Z_{(2)}^T)}^{-1}Z_{(2)}\Lambda)
            &=\textrm{rank}(I_p-\Lambda^{\frac{1}{2}} Z_{(2)}^T{(Z_{(2)}\Lambda Z_{(2)}^T)}^{-1}Z_{(2)}\Lambda^{\frac{1}{2}})\\
            &=p-n+2.\\
        \end{aligned}
    \end{equation}
    Hence
    \[
        T\sim\sum_{i=1}^{p-n+2} \lambda_{i}(\Lambda -\Lambda Z_{(2)}^T{(Z_{(2)}\Lambda Z_{(2)}^T)}^{-1}Z_{(2)}\Lambda) \chi^2_1
        \]
By Weyl's inequality, we have for $1\leq i\leq p-n+2$
\begin{equation}
    \lambda_i(\Lambda -\Lambda Z_{(2)}^T{(Z_{(2)}\Lambda Z_{(2)}^T)}^{-1}Z_{(2)}\Lambda)
    \leq \lambda_i(\Lambda),
\end{equation}
and
\begin{equation}
    \begin{aligned}
        &\lambda_i(\Lambda -\Lambda Z_{(2)}^T{(Z_{(2)}\Lambda Z_{(2)}^T)}^{-1}Z_{(2)}\Lambda)\\
        \geq& \lambda_{i+n-2}(\Lambda)+\lambda_{p-n+2}(-\Lambda Z_{(2)}^T{(Z_{(2)}\Lambda Z_{(2)}^T)}^{-1}Z_{(2)}\Lambda)\\
        =& \lambda_{i+n-2}.
    \end{aligned}
\end{equation}
Hence
\[
    \sum_{i=n-1}^p \lambda_i \chi^2_1\leq T\leq\sum_{i=1}^{p-n+2}\lambda_i \chi^2_1
    \]

Note that under condition \({\textrm{tr}\Sigma^4}/{{(\textrm{tr}\Sigma^2)}^2} \to 0\), we have by Liapounoff central limit theorem that
\[
    \frac{\sum_{i=1}^p \lambda_i \chi^2_1-\textrm{tr}\Sigma_X}{
        \sqrt{\textrm{tr}(\Sigma_X^2)}
    }\xrightarrow{\mathcal{L}}N(0,1).
    \]
And 
    \begin{equation}\label{xiaoO}
    \frac{T-\textrm{tr}\Sigma_X}{\sqrt{\textrm{tr}(\Sigma_X^2)}}-
    \frac{\sum_{i=1}^p \lambda_i \chi^2_1-\textrm{tr}\Sigma_X}{\sqrt{\textrm{tr}(\Sigma_X^2)}}
    =\frac{T-\sum_{i=1}^p \lambda_i \chi^2_1}{\sqrt{\textrm{tr}(\Sigma_X^2)}},
    \end{equation}
    To prove $\eqref{xiaoO}\xrightarrow{P}0$, we only need to prove
\[
    \textrm{E}\Big(\frac{\sum_{i=1}^{n-2} \lambda_i \chi^2_1}{
        \sqrt{\textrm{tr}(\Sigma_X^2)}}\Big)\to 0,
    \]
that is
    \begin{equation}\label{tiaojian}
    \textrm{E}\Big(\frac{\sum_{i=1}^{n-2} \lambda_i}{\sqrt{\textrm{tr}(\Sigma_X^2)}}\Big)\to 0.
    \end{equation}
If $\lambda_i$'s are bounded below and above, then~\eqref{tiaojian} is equivalent to

    \begin{equation}\label{npOrder}
        n/\sqrt{p}\to 0,
    \end{equation}
    or $p/n^2 \to \infty$. We thus obtain the following theorem.


\begin{theorem}
    Suppose
    \[
    \textrm{E}\Big(\frac{\sum_{i=1}^{n-2} \lambda_i}{\sqrt{\textrm{tr}(\Sigma_X^2)}}\Big)\to 0,
        \] 
    and
    \[
        \frac{\textrm{tr}\Sigma^4}{{(\textrm{tr}\Sigma^2)}^2} \to 0.
        \]
    Then under null hypotheses, we have
    \[
    \frac{T-\textrm{tr}\Sigma_X}{
        \sqrt{\textrm{tr}(\Sigma_X^2)}
    }\xrightarrow{\mathcal{L}}N(0,1).
        \]
\end{theorem}
\section{Simulation Results}
% latex table generated in R 3.3.1 by xtable 1.8-2 package
% Fri Nov  4 20:30:26 2016
\begin{table}[ht]
    \centering
    \begin{tabular}{rrrrr}
          \hline
          $n$ & $p$ & $|\beta|^2$ & Chen & New \\ 
            \hline
        40 & 310 & 0.00 & 0.05 & 0.06 \\ 
          40 & 310 & 0.04 & 0.05 & 1.00 \\ 
            80 & 550 & 0.00 & 0.05 & 0.01 \\ 
              80 & 550 & 0.04 & 0.05 & 1.00 \\ 
                             \hline
    \end{tabular}
    \caption{Non-sparse case, $T=1$}
\end{table}

\begin{table}[ht]
\centering
\begin{tabular}{rrrrr}
      \hline
          $n$ & $p$ & $|\beta|^2$ & Chen & New \\ 
        \hline
    40 & 310 & 0.00 & 0.05 & 0.00 \\ 
      40 & 310 & 0.04 & 0.35 & 1.00 \\ 
        80 & 550 & 0.00 & 0.05 & 0.00 \\ 
          80 & 550 & 0.04 & 0.29 & 1.00 \\ 
                         \hline
\end{tabular}
    \caption{Non-sparse case, $T=10$}
\end{table}

\begin{table}[ht]
\centering
\begin{tabular}{rrrrr}
      \hline
          $n$ & $p$ & $|\beta|^2$ & Chen & New \\ 
        \hline
        40 & 310 & 0.00 & 0.05 & 0.00 \\ 
          40 & 310 & 0.04 & 0.88 & 1.00 \\ 
            80 & 550 & 0.00 & 0.05 & 0.00 \\ 
              80 & 550 & 0.04 & 1.00 & 1.00 \\ 
                         \hline
\end{tabular}
    \caption{Non-sparse case, $T=20$}
\end{table}

% latex table generated in R 3.3.1 by xtable 1.8-2 package
% Fri Nov  4 21:17:00 2016
\begin{table}[ht]
    \centering
    \begin{tabular}{rrrrr}
          \hline
          $n$ & $p$ & $|\beta|^2$ & Chen & New \\ 
            \hline
                             40 & 310 & 0.00 & 0.05 & 0.06 \\ 
                               40 & 310 & 0.04 & 0.05 & 0.88 \\ 
                                 80 & 550 & 0.00 & 0.05 & 0.01 \\ 
                                   80 & 550 & 0.04 & 0.05 & 0.98 \\ 
                             \hline
    \end{tabular}
    \caption{Sparse case, $T=1$}
\end{table}

% latex table generated in R 3.3.1 by xtable 1.8-2 package
% Fri Nov  4 21:17:00 2016
\begin{table}[ht]
    \centering
    \begin{tabular}{rrrrr}
          \hline
          $n$ & $p$ & $|\beta|^2$ & Chen & New \\ 
            \hline
            40 & 310 & 0.00 & 0.05 & 0.00 \\ 
              40 & 310 & 0.04 & 0.06 & 0.20 \\ 
                80 & 550 & 0.00 & 0.05 & 0.00 \\ 
                  80 & 550 & 0.04 & 0.06 & 0.38 \\ 
                             \hline
    \end{tabular}
    \caption{Sparse case, $T=10$}
\end{table}


% latex table generated in R 3.3.1 by xtable 1.8-2 package
% Fri Nov  4 21:37:08 2016
\begin{table}[ht]
    \centering
    \begin{tabular}{rrrrr}
          \hline
          $n$ & $p$ & $|\beta|^2$ & Chen & New \\ 
            \hline
            40 & 310 & 0.00 & 0.05 & 0.00 \\ 
              40 & 310 & 0.04 & 0.12 & 0.04 \\ 
                80 & 550 & 0.00 & 0.05 & 0.00 \\ 
                  80 & 550 & 0.04 & 0.13 & 0.00 \\ 
                             \hline
    \end{tabular}
    \caption{Sparse case, $T=20$}
\end{table}

\section*{References}
\bibliography{mybibfile}

\end{document}
