

Next we derive another form of of $T$. We follow the similar technique of Hotelling's $T^2$.


Let $R$ be an $(n-1)\times (n-1)$ orthogonal matrix satisfies 
\[
    R\tilde{y}=
    \begin{pmatrix}
        \|\tilde{y}\|\\
        0\\
        \ldots\\
        0
    \end{pmatrix}.
    \]
%Note that $R$ is independent of $\tilde{X}$ under null hypotheses since it only depends on $\tilde{y}$.
We can write
    \begin{equation}\label{Tformula}
        \begin{aligned}
            T&=\frac{\|\tilde{y}\|^2
        }{\tilde{y}R^T{(R\tilde{X}^T\tilde{X}R^T)}^{-1}R\tilde{y}}\\
            %&={(R\tilde{X}^T \tilde{X} R^T)}_{11\cdot 2}.
        \end{aligned}
    \end{equation}
Denote by $B=R\tilde{X}^T \tilde{X} R^T$, then
\[
    T=\frac{1}{{(B^{-1})}_{11}}.
    \]

Let
\[
    B=\begin{pmatrix} 
        b_{11} & b_{(1)}^T\\
        b_{(1)} & B_{22}
    \end{pmatrix},
    \]

and apply the matrix inverse formula, we have ${(B^{-1})}_{11}=1/(b_{11}-b_{(1)}^T B_{22}^{-1}b_{(1)})$. Hence

\[
   T= b_{11}-b_{(1)}^T B_{22}^{-1}b_{(1)}.
    \]
\section{Asymptotic distribution}
Note that conditioning on $\tilde{y}$, $R$ is a constant orthogonal matrix.
And $\tilde{y}$ is independent of $\tilde{X}$ under null hypotheses.
So  $B|\tilde{y}$ has the same distribution with $\tilde{X}^T \tilde{X}$ under null hypotheses.
Hence $B$ is independent of $\tilde{y}$ and can be written as
    \begin{equation}\label{Xdis}
    B=\sum_{i=1}^p \lambda_i z_i z_i^T
    \end{equation}
where $z_i$'s are i.i.d.\ $n-1$ dimensional random vectors  distributed as $N(0,I_{n-1})$, $\lambda_1\geq \lambda_2\ldots \geq \lambda_p>0$ are eigenvalues of $\Sigma_X$. 
Denote by $\Lambda=\textrm{diag} (\lambda_1,\ldots,\lambda_p)$, $Z=(Z_1,\ldots,Z_p)$. Let $Z_{(1)}$ and $Z_{(2)}$ be the first $1$ row and last $n-2$ rows of $Z$, that is
\[
    Z=\begin{pmatrix} 
        Z_{(1)}\\
        Z_{(2)}
    \end{pmatrix}.
    \]
Then
\begin{equation}
    \begin{aligned}
        B&=Z\Lambda Z^T\\
        &=\begin{pmatrix}
            Z_{(1)}\Lambda Z_{(1)}^T & Z_{(1)}\Lambda Z_{(2)}^T\\
            Z_{(2)}\Lambda Z_{(1)}^T & Z_{(2)}\Lambda Z_{(2)}^T\\
        \end{pmatrix}.
    \end{aligned}
\end{equation}
Hence

\begin{equation}
    \begin{aligned}
        T&=Z_{(1)}\Lambda Z_{(1)}^T-Z_{(1)}\Lambda Z_{(2)}^T{(Z_{(2)}\Lambda Z_{(2)}^T)}^{-1}Z_{(2)}\Lambda Z_{(1)}^T\\
        &=Z_{(1)}\big(\Lambda -\Lambda Z_{(2)}^T{(Z_{(2)}\Lambda Z_{(2)}^T)}^{-1}Z_{(2)}\Lambda \big)Z_{(1)}^T.\\
    \end{aligned}
\end{equation}
But
    \begin{equation}
        \begin{aligned}
    \textrm{rank}(\Lambda Z_{(2)}^T{(Z_{(2)}\Lambda Z_{(2)}^T)}^{-1}Z_{(2)}\Lambda)
            &=\textrm{rank}(\Lambda^{\frac{1}{2}} Z_{(2)}^T{(Z_{(2)}\Lambda Z_{(2)}^T)}^{-1}Z_{(2)}\Lambda^{\frac{1}{2}})\\
            &=\textrm{rank}(I_{n-2})=n-2,\\
        \end{aligned}
    \end{equation}
and
    \begin{equation}
        \begin{aligned}
    \textrm{rank}(\Lambda-\Lambda Z_{(2)}^T{(Z_{(2)}\Lambda Z_{(2)}^T)}^{-1}Z_{(2)}\Lambda)
            &=\textrm{rank}(I_p-\Lambda^{\frac{1}{2}} Z_{(2)}^T{(Z_{(2)}\Lambda Z_{(2)}^T)}^{-1}Z_{(2)}\Lambda^{\frac{1}{2}})\\
            &=p-n+2.\\
        \end{aligned}
    \end{equation}
    Hence
    \[
        T\sim\sum_{i=1}^{p-n+2} \lambda_{i}(\Lambda -\Lambda Z_{(2)}^T{(Z_{(2)}\Lambda Z_{(2)}^T)}^{-1}Z_{(2)}\Lambda) \chi^2_1
        \]
By Weyl's inequality, we have for $1\leq i\leq p-n+2$
\begin{equation}
    \lambda_i(\Lambda -\Lambda Z_{(2)}^T{(Z_{(2)}\Lambda Z_{(2)}^T)}^{-1}Z_{(2)}\Lambda)
    \leq \lambda_i(\Lambda),
\end{equation}
and
\begin{equation}
    \begin{aligned}
        &\lambda_i(\Lambda -\Lambda Z_{(2)}^T{(Z_{(2)}\Lambda Z_{(2)}^T)}^{-1}Z_{(2)}\Lambda)\\
        \geq& \lambda_{i+n-2}(\Lambda)+\lambda_{p-n+2}(-\Lambda Z_{(2)}^T{(Z_{(2)}\Lambda Z_{(2)}^T)}^{-1}Z_{(2)}\Lambda)\\
        =& \lambda_{i+n-2}.
    \end{aligned}
\end{equation}
Hence
\[
    \sum_{i=n-1}^p \lambda_i \chi^2_1\leq T\leq\sum_{i=1}^{p-n+2}\lambda_i \chi^2_1
    \]

Note that under condition \({\textrm{tr}\Sigma^4}/{{(\textrm{tr}\Sigma^2)}^2} \to 0\), we have by Liapounoff central limit theorem that
\[
    \frac{\sum_{i=1}^p \lambda_i \chi^2_1-\textrm{tr}\Sigma_X}{
        \sqrt{\textrm{tr}(\Sigma_X^2)}
    }\xrightarrow{\mathcal{L}}N(0,1).
    \]
And 
    \begin{equation}\label{xiaoO}
    \frac{T-\textrm{tr}\Sigma_X}{\sqrt{\textrm{tr}(\Sigma_X^2)}}-
    \frac{\sum_{i=1}^p \lambda_i \chi^2_1-\textrm{tr}\Sigma_X}{\sqrt{\textrm{tr}(\Sigma_X^2)}}
    =\frac{T-\sum_{i=1}^p \lambda_i \chi^2_1}{\sqrt{\textrm{tr}(\Sigma_X^2)}},
    \end{equation}
    To prove $\eqref{xiaoO}\xrightarrow{P}0$, we only need to prove
\[
    \textrm{E}\Big(\frac{\sum_{i=1}^{n-2} \lambda_i \chi^2_1}{
        \sqrt{\textrm{tr}(\Sigma_X^2)}}\Big)\to 0,
    \]
that is
    \begin{equation}\label{tiaojian}
    \textrm{E}\Big(\frac{\sum_{i=1}^{n-2} \lambda_i}{\sqrt{\textrm{tr}(\Sigma_X^2)}}\Big)\to 0.
    \end{equation}
If $\lambda_i$'s are bounded below and above, then~\eqref{tiaojian} is equivalent to

    \begin{equation}\label{npOrder}
        n/\sqrt{p}\to 0,
    \end{equation}
    or $p/n^2 \to \infty$. We thus obtain the following theorem.


\begin{theorem}
    Suppose
    \[
    \textrm{E}\Big(\frac{\sum_{i=1}^{n-2} \lambda_i}{\sqrt{\textrm{tr}(\Sigma_X^2)}}\Big)\to 0,
        \] 
    and
    \[
        \frac{\textrm{tr}\Sigma^4}{{(\textrm{tr}\Sigma^2)}^2} \to 0.
        \]
    Then under null hypotheses, we have
    \[
    \frac{T-\textrm{tr}\Sigma_X}{
        \sqrt{\textrm{tr}(\Sigma_X^2)}
    }\xrightarrow{\mathcal{L}}N(0,1).
        \]
\end{theorem}
